\section{Части речи}\label{sec:speech_parts}


\subsection{Базовое использование}

Все команды данной группы имеют следующую сигнатуру:
\ExplSyntaxOn
\begin{tcolorbox}
    \rsModifier[cmd] \rsColon{} \rsCode{<name>}
    \rsKwargs[\textasteriskcentered]
    \rsReq{ \rsArg[<слово:tl>] }
    \rsOpt{ \rsArg[<анализ:tl>] }
\end{tcolorbox}
\ExplSyntaxOff
где \rsCode{<name>} может быть одним из следующих \rsModifier[значений]:

\begin{table}[ht!]
    \centering
    \begin{tabular}{@{}llll@{}}
        \toprule

        \rsModifier[rsNoun]          & существительное
                                     &
        \rsModifier[rsVerb]          & глагол
        \\\midrule

        \rsModifier[rsAdverb]        & наречие
                                     &
        \rsModifier[rsProposition]   & предлог
        \\\midrule

        \rsModifier[rsConjunction]   & союз
                                     &
        \rsModifier[rsPronoun]       & местоимение
        \\\midrule

        \rsModifier[rsAdjective]     & прилагательное
                                     &
        \rsModifier[rsParticle]      & частица
        \\\midrule

        \rsModifier[rsInterjection]  & междометие
                                     &
        \rsModifier[rsNumeral]       & числительное
        \\\midrule


        \rsModifier[rsParticiple]    & причастие
                                     &
        \rsModifier[rsTransgressive] & деепричастие

        \\\bottomrule
    \end{tabular}
    \caption{Команды для частей речи}
    \label{tab:pos-commands}
\end{table}

\begin{tnote}
    Команды данного раздела чувствительны к опции \rsCode{phantom}.
\end{tnote}

Базовые сценарии использования команд для частей речи на примере \rsModifier[rsNoun] представлены в таблице ниже.

\ExplSyntaxOn
\begin{table}[ht!]
    \centering
    \begin{tabular}{@{}ll@{}}
        \toprule

        \rsModifier*[rsNoun]
        \rsReq{ \rsArg[существительное] }
         &
        \rsNoun{существительное}
        \\\midrule

        \rsModifier*[rsNoun]
        \rsReq{ \rsArg[существительное] }
        \rsOpt{ \rsArg[ср.р., им.п.] }
         &
        \rsNoun{существительное}[ ср.р., им.п. ]
        \\\midrule

        \rsModifier*[rsNoun]
        \textasteriskcentered{}
        \rsReq{ \rsArg[существительное] }
         &
        \rsNoun*{существительное}
        \\\midrule

        \rsModifier*[rsNoun]
        \textasteriskcentered{}
        \rsReq{ \rsArg[существительное] }
        \rsOpt{ \rsArg[СУЩ.] }
         &
        \rsNoun*{существительное}[СУЩ.]
        \\\midrule

        \rsModifier*[rsNoun]
        \textasteriskcentered{}
        \rsReq{ \rsArg[существительное] }
        \rsOpt{ \rsArg[СУЩ., ср.р., им.п] }
         &
        \rsNoun*{существительное}[СУЩ., ср.р., им.п]
        \\\midrule
        \bottomrule
    \end{tabular}
    \caption{Использование~команд~частей~речи}
\end{table}
\ExplSyntaxOff


\subsection{Сокращённые названия частей речи}


\subsubsection{Базовое использование}\label{subsubsec:pos-basic}

Для отображения сокращённых названий частей речи в тексте предназначены команды со следующей сигнатурой:
\ExplSyntaxOn
\begin{tcolorbox}
    \rsModifier[cmd] \rsColon{} \rsCode{<name> Acr}
\end{tcolorbox}
\ExplSyntaxOff
где \rsCode{<name>} принимает значения команд из таблицы «\nameref{tab:pos-commands}».

Например, \rsCode{\rsModifier*[rsNounAcr]\rsReq{}} даст «\rsNounAcr{}», использование фигурных скобок необходимо для установки пробела после сокращения.


\subsubsection{Локализация стандартных сокращений частеречных команд}

Пакет \rsCode{rslingu} поддерживает локализацию для сокращений частей речи во всех командах из
таблицы «\nameref{tab:pos-commands}» через модификатор \rsModifier[rsSetLanguage]:
\ExplSyntaxOn
\begin{tcolorbox}
    \rsCode{
        \rsModifier[cmd] \rsColon rsSetLanguage
        \rsReq{ \rsArg[<язык: tl>] }
    }
\end{tcolorbox}
\ExplSyntaxOff

На данный момент доступны следующие языки:
\begin{table}[ht!]
    \centering
    \begin{tabular}{@{}ll@{}}
        \toprule

        Язык       & \rsArg[язык] \\
        \midrule

        русский    & russian      \\
        украинский & ukranian     \\

        \bottomrule
    \end{tabular}
    \caption{Доступные языки для локализации частеречных команд}
\end{table}

Языком по умолчанию является русский. Со списком стандартных сокращений для разных языков
можно ознакомиться в приложении \ref{appendix:pos-contractions}
«\nameref{appendix:pos-contractions}».


\begin{figure}[H]
    \centering
    \begin{minipage}[c]{0.5\textwidth}
        \begin{Latexcode}
            \rsSetLanguage{russian}  % По умолчанию.
            \rsParticiple{Уставшая} \rsNoun{мама}
            \rsVerb{мыла} \rsNoun{раму}
            \rsNoun{вечером}.

            \rsSetLanguage{ukranian}
            \rsVerb{Мріють} \rsNoun{крилами}
            \rsPreposition{з} \rsNoun{туману}
            \rsNoun{лебеді} \rsAdjective{рожеві}.
        \end{Latexcode}
    \end{minipage}
    \hfill
    \begin{minipage}[c]{0.4\textwidth}
        \small
        \rsSetLanguage{russian}  % Язык по умолчанию.
        \rsParticiple{Уставшая}
        \rsNoun{мама} \rsVerb{мыла} \rsNoun{раму} \rsNoun{вечером}.
        \vspace*{\baselineskip}

        \rsSetLanguage{ukranian}
        \rsVerb{Мріють} \rsNoun{крилами} \rsPreposition{з}
        \rsNoun{туману} \rsNoun{лебеді} \rsAdjective{рожеві}.
    \end{minipage}

    \caption{Пример смены языка частеречных сокращений}
\end{figure}