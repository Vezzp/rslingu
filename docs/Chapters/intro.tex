\section{Условные обозначения}

Все макросы, или текстовые модификаторы текста, пакета \texttt{rslingu} в данном руководстве
задаются следующим образом:
\ExplSyntaxOn
\begin{tcolorbox}
    \rsCode {
        \rsModifier[modifier_type]
        \rsSpace{}\rsColon{}\rsSpace{}
        modifier_name
        \rsSpace{}
        \rsOpt\rsKwargs
        \rsSpace{}
        < [, \{ >
                \rsSpace{}
                \rsArg{}:type_hint
                \rsSpace{}< ], \} >
        \rsSpace{}... \\

        \rsTab{} \rsKwargs<> \rsSpace{}
        kwarg_name \rsSpace{} \rsColon{} \rsSpace{}type_hint = kwarg_default_value \rsSpace{} ...
    }
\end{tcolorbox}
\ExplSyntaxOff

\begin{itemize}
    \item \rsModifier[modifier\_type] — тип текстового модификатора, команда (\rsModifier[cmd]) или окружение (\rsModifier[env]).

    \item \rsKwargs{} — ненулевое количество именованных необязательных аргументов.

    \item
          \rsArg{} — ненулевое количество обязательных и/или необязательных аргументов.
\end{itemize}

По возможности, у каждого агрумента будут указаны подсказки об обозначении аргумента и
его типе (\texttt{type\_hint}), который может принимать следующие значения:

\begin{itemize}
    \item \texttt{tl}, от англ. \textit{token list}, — произвольный набор символов, обрабатывается целиком.

    \item \texttt{clist}, от англ. \textit{comma list}, — набор произвольных символов, разделённых запятой, где
          каждый поднабор до запятой обрабатывается независимо.

    \item \texttt{bool} — логический тип, может принимать значения «true» или «false».
\end{itemize}
