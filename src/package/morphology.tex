% Цвет морфем, задаваемый внутри команд.
\tl_new:N \l__rslingu_morpheme_color_tl
\cs_new_protected:Nn \l__rslingu_morpheme_set_color_cs:n {
    \tl_set:Nn \l__rslingu_morpheme_color_tl {
        \bool_if:nTF
            { \l__rslingu_morpheme_color_bool || \l__rslingu_color_bool }
            { rs #1 color }
            { black }
    }
}
 
% Высота знаков морфем (за исключением основы).
\dim_const:Nn \c__rslingu_morpheme_sign_height_dim {3pt}

% Ограничение длины знака морфемы по сравнению с длиной морфемы (за исключением основы).
% NB: Необходимо для избегания наложения знака одной морфемы на знак другой.
\dim_const:Nn \c__rslingu_morpheme_sign_x_pad_dim {1pt}

% Расстояние от нижней части знака морфем до высшей части морфемы (за исключением основы).
% NB: Расстояние считается как если бы в морфеме была буква «б».
\dim_const:Nn \c__rslingu_morpheme_sign_y_pad_dim {0pt}

% Толщина линий знака морфем (включая основу).
\dim_const:Nn \c__rslingu_morpheme_sign_line_width_dim {1.25pt}

% Высота знака основы слова.
\dim_const:Nn \c__rslingu_morpheme_base_sign_height_dim {2pt}

% Ограничение длины знака основы слова по сравнению с длиной морфемы.
% NB: Необходимо для избегания наложения знака одной морфемы на знак другой.
\dim_const:Nn \c__rslingu_morpheme_base_sign_x_pad_dim {1pt}

% Расстояние от нижней части основы до знака основы.
\dim_const:Nn \c__rslingu_morpheme_base_sign_y_pad_dim {1pt}

% TikZ-стиль для оформления знаков морфем.
\tikzset{
    rsMorphemeSign/.style={
        line~width=\c__rslingu_morpheme_sign_line_width_dim,
    }
}

% Команда для выбора содержимого морфемы в завимости от значения флага сокрытия `phantom`.
\cs_new_protected:Nn \l__rslingu_morpheme_place_content_cs:n {
    \bool_if:nTF
    {
        \l__rslingu_morpheme_phantom_bool
        ||
        \l__rslingu_phantom_bool
    }
    {
        \node[rsContentBox] (Baseline) {
            \phantom{#1} \c__rslingu_vmax_phantom_tl
        };
        \node at (Baseline.center) { \textbullet };
    }
    {
        \node[rsContentBox] (Baseline) { #1 \c__rslingu_vmax_phantom_tl };
    }
}

\cs_generate_variant:Nn \l__rslingu_morpheme_place_content_cs:n { V }




% +++++++++++++++++++++++++++++++++++++++++++++++++++++++++++++++++++++++++++++++++++
% ++++++++++++++++++++++++++++++++ MORPHEME COMMANDS ++++++++++++++++++++++++++++++++
% +++++++++++++++++++++++++++++++++++++++++++++++++++++++++++++++++++++++++++++++++++

% ======================================================
% ================ Prefix (приставка) ==================
% ======================================================


\cs_new_protected:Nn \l__rslingu_morpheme_prefix_cs:n {
    \group_begin:
        \clist_map_variable:nNn { #1 } { \l__item } {
            \tikz[baseline=(Baseline.base)] {
                \l__rslingu_morpheme_place_content_cs:V \l__item
                \l__rslingu_morpheme_set_color_cs:n { prefix }
                
                \draw[
                    rsMorphemeSign,
                    \l__rslingu_morpheme_color_tl,
                    transform~canvas={
                        yshift=\c__rslingu_morpheme_sign_y_pad_dim + \c__rslingu_morpheme_sign_height_dim
                    }
                ]
                let
                    \p1 = (Baseline.north~west),
                    \p2 = (Baseline.north~east)
                in
                    (\x1 + \c__rslingu_morpheme_sign_x_pad_dim, \y1 + \c__rslingu_morpheme_sign_y_pad_dim) --
                    (\x2 - \c__rslingu_morpheme_sign_x_pad_dim, \y2 + \c__rslingu_morpheme_sign_y_pad_dim) --
                    ++(0, -\c__rslingu_morpheme_sign_height_dim);
            }
        }
    \group_end:
}

\g__rslingu_morphology_public_codegen_cs:n { prefix }


% ======================================================
% ======================================================


% ======================================================
% ================== Root (корень) =====================
% ======================================================


\cs_new_protected:Nn \l__rslingu_morpheme_root_cs:n {
    \group_begin:
        \clist_map_variable:nNn { #1 } { \l__item } {
            \tikz[baseline=(Baseline.base)] {
                \l__rslingu_morpheme_place_content_cs:V \l__item
                \l__rslingu_morpheme_set_color_cs:n { root }
                \draw[
                    rsMorphemeSign,
                    \l__rslingu_morpheme_color_tl,
                    transform~canvas={
                        yshift=\c__rslingu_morpheme_sign_y_pad_dim
                    }
                ]
                let
                    \p1 = (Baseline.north~west),
                    \p2 = (Baseline.north~east)
                in
                    (\x1 + \c__rslingu_morpheme_sign_x_pad_dim, \y1 + \c__rslingu_morpheme_sign_y_pad_dim)
                    parabola[bend~pos=.5] bend + (0, \c__rslingu_morpheme_sign_height_dim)
                    (\x2 - \c__rslingu_morpheme_sign_x_pad_dim, \y2 + \c__rslingu_morpheme_sign_y_pad_dim);
            }
        }
    \group_end:
}

\g__rslingu_morphology_public_codegen_cs:n { root }


% ======================================================
% ======================================================


% ======================================================
% ================= Suffix (суффикс) ===================
% ======================================================


\cs_new_protected:Nn \l__rslingu_morpheme_suffix_cs:n {
    \group_begin:
    \clist_map_variable:nNn { #1 } { \l__item } {
        \tikz[baseline=(Baseline.base)] {
            \l__rslingu_morpheme_place_content_cs:V \l__item
            \l__rslingu_morpheme_set_color_cs:n { suffix }
            \draw[
                rsMorphemeSign,
                \l__rslingu_morpheme_color_tl,
                transform~canvas={
                    yshift=\c__rslingu_morpheme_sign_y_pad_dim
                }
            ]
            let
                \p1 = (Baseline.north~west),
                \p2 = (Baseline.north~east)
            in
                (\x1 + \c__rslingu_morpheme_sign_x_pad_dim, \y1 + \c__rslingu_morpheme_sign_y_pad_dim) --
                (.5 *\x1 + .5 * \x2, \y2 + \c__rslingu_morpheme_sign_height_dim) --
                (\x2 - \c__rslingu_morpheme_sign_x_pad_dim, \y2 + \c__rslingu_morpheme_sign_y_pad_dim);
        }
    }
    \group_end:
}

\g__rslingu_morphology_public_codegen_cs:n { suffix }


% ======================================================
% ======================================================


% ======================================================
% ================= Postfix (постфикс) =================
% ======================================================


    \cs_new_protected:Nn \l__rslingu_morpheme_postfix_cs:n {
        \group_begin:
            \clist_map_variable:nNn { #1 } { \l__item } {
                \tikz[baseline=(Baseline.base)] {
                    \l__rslingu_morpheme_place_content_cs:V \l__item
                    \l__rslingu_morpheme_set_color_cs:n { postfix }
                    \draw[
                        rsMorphemeSign,
                        \l__rslingu_morpheme_color_tl,
                        transform~canvas={
                            yshift=\c__rslingu_morpheme_sign_y_pad_dim
                        }
                    ]
                    let
                        \p1 = (Baseline.north~west),
                        \p2 = (Baseline.north~east)
                    in
                        % Первый знак суффикса.
                        (\x1 + \c__rslingu_morpheme_sign_x_pad_dim, \y1 + \c__rslingu_morpheme_sign_y_pad_dim) --
                        (.5 *\x1 + .5 * \x2, \y2 + \c__rslingu_morpheme_sign_height_dim) --
                        (\x2 - \c__rslingu_morpheme_sign_x_pad_dim, \y2 + \c__rslingu_morpheme_sign_y_pad_dim)
                        % Второй знак суффикса.
                        (\x1 + \c__rslingu_morpheme_sign_x_pad_dim, \y1 + \c__rslingu_morpheme_sign_y_pad_dim + 3pt) --
                        (.5 *\x1 + .5 * \x2, \y2 + \c__rslingu_morpheme_sign_height_dim + 3pt) --
                        (\x2 - \c__rslingu_morpheme_sign_x_pad_dim, \y2 + \c__rslingu_morpheme_sign_y_pad_dim + 3pt);
                }
            }
        \group_end:
    }

    \g__rslingu_morphology_public_codegen_cs:n { postfix }


% -----------------------------------------------------
% -----------------------------------------------------


% ======================================================
% ================= Ending (окончание) =================
% ======================================================

    \NewDocumentCommand{\rsEnding} { O{} m } {
    \group_begin:
        \keys_set_known:nn { rslingu/morphemes/keys } { phantom=false, color=false, #1 }
        % Чтобы знак оконачения не прилипал слева. 
        \hspace{.75pt}

        \tikz[baseline=(Baseline.base)] {
        
            \bool_if:nTF
            {
                \l__rslingu_morpheme_phantom_bool
                ||
                \l__rslingu_phantom_bool
            }
            {
                \node[rsContentBox, outer~sep=1pt] (Baseline)
                {
                    \phantom{o} \c__rslingu_vmax_phantom_tl
                };
                \node at (Baseline.center) { \textbullet };
            }
            {
                \node[rsContentBox, outer~sep=1pt] (Baseline)
                {
                    \tl_if_blank:nTF
                    { #2 }
                    { \phantom{o} }
                    { #2 }
                    \c__rslingu_vmax_phantom_tl
                };
            }        
        
            \l__rslingu_morpheme_set_color_cs:n { ending }

            \draw[
                rsMorphemeSign,
                \l__rslingu_morpheme_color_tl,
                transform~canvas={
                    yshift=\c__rslingu_morpheme_sign_y_pad_dim
                },
            ]
            let
                \p1 = (Baseline.south~west),
                \p2 = (Baseline.north~east)
            in
                % NOTE: Нет контроля типа шрифта, то есть наклонном шрифте будет некрасиво.
                % По-хорошему, .75 — это мусор, и нужно действовать как-то иначе.
                (
                    \x1 - .75,
                    \y1 - \c__rslingu_morpheme_base_sign_height_dim
                )
                |-
                (
                    \x2 + .75,
                    \y2
                        + \c__rslingu_morpheme_sign_y_pad_dim
                        + \c__rslingu_morpheme_sign_height_dim
                )
                |-
                cycle;
        }

        % Чтобы знак оконачения не прилипал слева.
        \hspace{.75pt}
  \group_end:
    }

% -----------------------------------------------------
% -----------------------------------------------------


% ======================================================
% =================== Base (основа) ====================
% ======================================================

    \NewDocumentCommand{\rsBase}{ O{} m }{
        \group_begin:
            \keys_set_known:nn { rslingu/morphemes/keys } { color=false, #1 }
            \keys_set_known:nn { rslingu/morphemes/base/keys } { right=false, left=false, #1 }
            \tikz[baseline=(Baseline.base)] {
                \node[rsContentBox] (Baseline) {#2 \c__rslingu_below_vmax_phantom_tl};
                \l__rslingu_morpheme_set_color_cs:n { base }
                % «Foreach» — это «костыли».
                \foreach \n in {0}
                    \draw[
                        rsMorphemeSign,
                        \l__rslingu_morpheme_color_tl,,
                        transform~canvas={
                            yshift=-\c__rslingu_morpheme_base_sign_y_pad_dim - \c__rslingu_morpheme_base_sign_height_dim,
                        },
                    ]
                        let
                            \p1 = ([xshift=\c__rslingu_morpheme_base_sign_y_pad_dim]Baseline.south~west),
                            \p2 = ([xshift=-\c__rslingu_morpheme_base_sign_y_pad_dim - .5pt]Baseline.south~east)
                        in
                            % Правая часть основы.
                            \bool_if:NF \l__rslingu_morpheme_base_right_bool {
                                (\p1) -- ([yshift=\c__rslingu_morpheme_base_sign_height_dim]\p1) |-
                            }
                            (\p1) -- (\p2)
                            % Левая часть основы.
                            \bool_if:NF \l__rslingu_morpheme_base_left_bool {
                                -|
                                (\p2) -- ([yshift=\c__rslingu_morpheme_base_sign_height_dim]\p2)
                            };
            }
      \group_end:
    }

% -----------------------------------------------------
% -----------------------------------------------------



    % Публичная команда для морфемного анализа слова без разрывной основы.
    \NewDocumentCommand{\rsMorphemicAnalysis} { O{} m m m m m } {
        \rsBase[#1] {
            \rsPrefix[#1]{#2}
            \rsRoot[#1]{#3}
            \rsSuffix[#1]{#4}
        }
        \rsEnding[#1]{#5}
        \rsPostfix[#1]{#6}
    }
