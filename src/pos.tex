\ExplSyntaxOn

% Генерация списка доступных языков для частеречных сокращений и выставление
% языка по умолчанию.
% template: {{ pos_lang_codegen }}

\tl_new:N \g__rslingu_current_lang_tl
\tl_set_eq:NN \g__rslingu_current_lang_tl \c__rslingu_default_lang_tl

\NewDocumentCommand{\rsSetLanguage}{ m }{
    \seq_if_in:NnTF \c__rslingu_supported_lang_seq { #1 }
    {
        \tl_set:Nn \g__rslingu_current_lang_tl { #1 }
    }
    {
        \PackageWarning { rslingu } {
            Language~option~' #1 '~is~not~known,~choose~one~of~
            [\seq_use:Nn \c__rslingu_supported_lang_seq {,~}].~
            Using~' \c__rslingu_default_lang_tl '~option~as~the~default
        }
    }
}

\cs_new_protected:Nn \l_rslingu_pos_codegen_cs:n {

    % Сокращённое имя части речи. Например, если #1 == Noun, то сгенирируется
    % команда \rsNounAcr
    \exp_after:wN
    \NewDocumentCommand
    \cs:w
        rs
        \text_titlecase_first:n { #1 }
        Acr
    \cs_end:
    { }
    {
        \use:c {
            l__rslingu_syntax_
            #1
            _short_
            \tl_use:N \g__rslingu_current_lang_tl
            _tl
        }
    }
    

    
    % Обёртка для частей речи. Например, если #1 == Noun, то сгенирируется
    % команда \rsNoun
    \exp_after:wN
    \NewDocumentCommand
    \cs:w
        rs
        \text_titlecase_first:n { #1 }
    \cs_end:
    { s m O{} }
    {
    \group_begin:
        \textoverset
        {
            \tikz[baseline=(Aux.base)]
            {
                \node[rsContentBox] (Aux)
                {
                    \bool_if:nTF
                        {
                           (
                               \l__rslingu_phantom_bool
                                ||
                                \l__rslingu_syntax_phantom_bool
                            )
                        }
                        {
                            \phantom{##2}
                        }
                        {
                            ##2
                        }
                };
                
                \bool_if:nTF
                {
                    (
                        \l__rslingu_phantom_bool
                        ||
                        \l__rslingu_syntax_phantom_bool
                    )
                }
                { \node[rsContentBox] at (Aux.center) {\textbullet}; }
                { }
            }
        }
        {   
            \IfBooleanTF
            { ##1 }
            { }
            { \cs:w rs \text_titlecase_first:n { #1 } Acr \cs_end: }

            \tl_if_blank:nTF { ##3 } {  }
            {
                \IfBooleanTF{ ##1 } { } {,~}
                ##3            
            }
        }
        
    \group_end:
    }
}


\exp_after:wN
\NewDocumentCommand
\cs:w
    rsSetAcr
\cs_end: 
{ O{} m m }
{
    \tl_set:cn
    {
        l__rslingu_syntax_
        #2
        _short_
        \tl_if_empty:nTF { \tl_use:c g__rslingu_current_lang_tl } { #1 } { #1 }
        _tl
    }
    {
        #3
    }
}


\exp_after:wN
\NewDocumentCommand
\cs:w
    rsResetAcr
\cs_end:
{ O{} m }
{
    \tl_set_eq:cc
    {
        l__rslingu_syntax_
        #2
        _short_
        \tl_if_empty:nTF { #1 } { \g__rslingu_current_lang_tl } { #1 }
        _tl
    }
    {
        c__rslingu_syntax_
        #2
        _default_short_
        \tl_if_empty:nTF { #1 } { \g__rslingu_current_lang_tl } { #1 }
        _tl
    }
}


\exp_after:wN
\NewDocumentCommand
\cs:w
    rsShowAcr
\cs_end:
{ s O{} m }
{
    \IfBooleanTF{ #1 } {
        \tl_use:c {
            c__rslingu_syntax_
            #3
            _default_short_
            \tl_if_empty:nTF { #2 } { \g__rslingu_current_lang_tl } { #2 }
            _tl
        }
    } {
        \tl_use:c {
        l__rslingu_syntax_
        #3
        _short_
        \tl_if_empty:nTF { #2 } { \g__rslingu_current_lang_tl } { #2 }
        _tl
    }
    }
}


% Генерация констант для сокращений частей речи по умолчанию и пользовательских
% переменных.
% #1 — название части речи (`en_full_name`)
% #2 — название языка
% #3 — сокращение части речи
\cs_new_protected:Nn \l_set_pos_acr_codegen_cs:nnn {
    \tl_const:cn
    { c__rslingu_syntax_ #1 _default_short_ #2 _tl }
    { #3 }
    
    \tl_set_eq:cc
    { l__rslingu_syntax_ #1 _short_ #2 _tl }
    { c__rslingu_syntax_ #1 _default_short_ #2 _tl }
}

% Генерация частеречных сокращений для разных языков и публичных команд дл них.
% template: {{ pos_codegen }}

\ExplSyntaxOff