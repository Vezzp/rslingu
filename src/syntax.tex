% Цвета синтаксических подчёркиваний.
\definecolor{rsSubjectColor}{HTML}{673ab7}
\definecolor{rsPredicateColor}{HTML}{e81e62}
\definecolor{rsAttributeColor}{HTML}{2196f3}
\definecolor{rsAdverbialColor}{HTML}{009688}
\definecolor{rsObjectColor}{HTML}{ffa500}




% Цвет синтаксических подчёркиваний.
\tl_new:N \l__rslingu_syntax_color_tl
\cs_new_protected:Nn \l__rslingu_syntax_set_color_cs:n
{
    \tl_set:Nn \l__rslingu_syntax_color_tl
    {
        \bool_if:nTF
            { \l__rslingu_syntax_color_bool || \l__rslingu_color_bool}
            { #1 } 
            { black }
    }
}

\dim_const:Nn \c__rslingu_syntax_uline_shift_dim { 2pt }
\dim_const:Nn \c__rslingu_syntax_uline_width_dim { 1.25pt }

\tikzset{
    rsUline/.style={
        line~width=\c__rslingu_syntax_uline_width_dim,
    },
}





\cs_generate_variant:Nn \seq_set_split:Nnn { NnV }

% Флаг для постановки пробела между однородными членами, если они есть.
\bool_new:N \l__place_spacing_bool

\cs_new_protected:Nn \l__rslingu_syntax_place_content_cs:n
{
    \group_begin:

    \node[rsContentBox] (Baseline) {

        % Разделяем строку по запятым.
        \clist_map_variable:nNn { #1 } \l__homogeneous_item_tl {

            % Проверяем, нужно ли ставить пробел между однородными членами.
            \bool_if:nTF \l__place_spacing_bool {
                {~}
            } {
                \bool_set_true:N \l__place_spacing_bool
            }

            % Разделяем по знаку равно.
            \StrCut{ \l__homogeneous_item_tl } { = } \l__word_tl \l__aux_info_tl

            % Разделяем по знаку плюса.
            \seq_set_split:NnV \l__aux_info_seq { + } { \l__aux_info_tl }
            
            % Нужно ли использовать фантомный вариант.
            \bool_set:Nn \l_tmpa_bool {
                (
                    \l__rslingu_syntax_phantom_bool
                    ||
                    \l__rslingu_phantom_bool
                )
            }

            \tikz[baseline=(Aux.base)] {
                \node[
                    rsContentBox,
                    label={
                        [
                            inner~sep=0pt,
                            outer~sep=0pt,
                            font=\tiny\itshape,
                            yshift=.5pt
                        ]
                        % Печать дополнительной информации, содержащейся в `l__aux_info_seq`, если та непуста.
                        \seq_if_empty:NTF \l__aux_info_seq
                            { }
                            {
                                \seq_use:Nn \l__aux_info_seq {,~}
                            }
                    }
                ] (Aux) {
                    \bool_if:nTF { \l_tmpa_bool }
                    {
                        \phantom{\l__word_tl}
                    }
                    {
                        \l__word_tl
                    }

                    \c__rslingu_above_vmax_phantom_tl
                };

                \bool_if:nTF { \l_tmpa_bool } {
                    \node[rsContentBox] at (Aux.center) {\textbullet};
                } {}
            }
        }

        % Выравнивание по высоте внутри блока `Baseline`.
        \c__rslingu_below_vmax_phantom_tl
    };

    \bool_set_false:N \l__place_spacing_bool
    \group_end:
}






% ==============================================
% =========== Subject (подлежащее) =============
% ==============================================



\NewDocumentCommand \rsSubject { O{} m }
{
    \group_begin:
        \keys_set_known:nn
            { rslingu/syntax/keys }
            { color=false, phantom=false, #1 }
        
        \tikz[baseline=(Baseline.base)] {
            \l__rslingu_syntax_set_color_cs:n { rsSubjectColor }
            
            \node[rsContentBox] (Baseline)
            { #2 \c__rslingu_below_vmax_phantom_tl };
            
            \draw
            [
                rsUline,
                \l__rslingu_syntax_color_tl
            ]
            ([yshift=-\c__rslingu_syntax_uline_shift_dim]Baseline.south~west)
            --
            ([yshift=-\c__rslingu_syntax_uline_shift_dim]Baseline.south~east);
        }
    \group_end:
}





% % ==============================================
% % =========== Predicate (сказуемое) ============
% % ==============================================



\NewDocumentCommand \rsPredicate { O{} m } {
    \group_begin:
        \keys_set_known:nn
            { rslingu/syntax/keys }
            { color=false, phantom=false, #1 }

        \tikz[baseline=(Baseline.base)] {
            \l__rslingu_syntax_set_color_cs:n { rsPredicateColor }

            \node[rsContentBox] (Baseline)
            { #2 \c__rslingu_below_vmax_phantom_tl };

            \draw[
                rsUline,
                \l__rslingu_syntax_color_tl
            ]
                ([yshift=-\c__rslingu_syntax_uline_shift_dim]Baseline.south~west) --
                ([yshift=-\c__rslingu_syntax_uline_shift_dim]Baseline.south~east);

            \draw[
                rsUline,
                \l__rslingu_syntax_color_tl
            ]
                ([yshift=-\c__rslingu_syntax_uline_shift_dim - 2pt]Baseline.south~west) --
                ([yshift=-\c__rslingu_syntax_uline_shift_dim - 2pt]Baseline.south~east);
        }
    \group_end:
}







% % ==============================================
% % ======== Adverbial (обстоятельство) ========
% % ==============================================



\NewDocumentCommand \rsAdverbial { O{} m } {
    \group_begin:
        \keys_set_known:nn
            { rslingu/syntax/keys }
            { color=false, phantom=false, #1 }

        \tikz[baseline=(Baseline.base)] {
            \l__rslingu_syntax_set_color_cs:n { rsAdverbialColor }

            \node[rsContentBox] (Baseline)
            { #2 \c__rslingu_below_vmax_phantom_tl };
            
            \draw[
                rsUline,
                \l__rslingu_syntax_color_tl,
                dash~pattern={on 5pt off 2pt on 1.5pt off 2pt},
            ]
                ([yshift=-\c__rslingu_syntax_uline_shift_dim]Baseline.south~west) --
                ([yshift=-\c__rslingu_syntax_uline_shift_dim]Baseline.south~east);
        }
    \group_end:
}






% % ==============================================
% % ======== Attribute (определение) =============
% % ==============================================



\NewDocumentCommand{\rsAttribute} { O{} m } {
    \group_begin:
        \keys_set_known:nn { rslingu/syntax/keys } { color=false, phantom=false, #1 }

        \tikz[baseline=(Baseline.base)] {
            \l__rslingu_syntax_set_color_cs:n { rsAttributeColor }

            \node[rsContentBox] (Baseline)
            { #2 \c__rslingu_below_vmax_phantom_tl };

            \draw[
                rsUline,
                \l__rslingu_syntax_color_tl,
                decorate,
                decoration={
                    complete~sines,
                    segment~length=2.75pt,
                    amplitude=1.75,
                    mirror,
                    start~up,
                    end~down
                }
            ]
                ([yshift=-\c__rslingu_syntax_uline_shift_dim]Baseline.south~west) --
                ([yshift=-\c__rslingu_syntax_uline_shift_dim]Baseline.south~east);
        }
    \group_end:
}






% % ==============================================
% % ========== Object (обстоятельство) ===========
% % ==============================================



\NewDocumentCommand \rsObject { O{} m } {
    \group_begin:
        \keys_set_known:nn { rslingu/syntax/keys } { color=false, phantom=false, #1 }

        \tikz[baseline=(Baseline.base)] {
            \l__rslingu_syntax_set_color_cs:n { rsObjectColor }

            \node[rsContentBox] (Baseline)
            { #2 \c__rslingu_below_vmax_phantom_tl };

            \draw[
                rsUline,
                \l__rslingu_syntax_color_tl,
                dash~pattern={on 5pt off 2pt on 1pt off 0pt},
            ]
                ([yshift=-\c__rslingu_syntax_uline_shift_dim]Baseline.south~west) --
                ([yshift=-\c__rslingu_syntax_uline_shift_dim]Baseline.south~east);
        }
    \group_end:
}






% % ==============================================
% % =============== Части речи ===================
% % ==============================================




\cs_new_protected:Nn \l__rslingu_syntax_make_speech_part_cmd_cs:n {

    % Сокращённое имя части речи. Например, если #1 == Noun, то сгенирируется
    % команда \rsNounAcrS
    \exp_after:wN
    \NewDocumentCommand
    \cs:w rs #1 Acr \cs_end:
    { }
    {
        \use:c {
            l__rslingu_syntax_
            \str_lowercase:f #1
            _short_
            \tl_use:N \g__rslingu_current_lang_tl
            _tl
        }
    }
    
    \exp_after:wN
    \NewDocumentCommand
    \cs:w rs Set #1 Acr \cs_end: 
    { }
    {
        \tl_set:cn 
        {
            l__rslingu_syntax_
            \str_lowercase:f #1
            _short_
            \tl_use:N \g__rslingu_current_lang_tl
            _tl
        }
    }
    
    % Обёртка для частей речи. Например, если #1 == Noun, то сгенирируется
    % команда \rsNoun
    \exp_after:wN
    \NewDocumentCommand
    \cs:w rs #1 \cs_end:
    { s m O{} }
    {
    \group_begin:
        \textoverset
        {
            \tikz[baseline=(Aux.base)]
            {
                \node[rsContentBox] (Aux)
                {
                    \bool_if:nTF
                        {
                           (
                                \l__rslingu_phantom_bool
                                ||
                                \l__rslingu_syntax_phantom_bool
                            )
                        }
                        {
                            \phantom{##2}
                        }
                        {
                            ##2
                        }
                };
                
                \bool_if:nTF
                {
                    (
                        \l__rslingu_phantom_bool
                        ||
                        \l__rslingu_syntax_phantom_bool
                    )
                }
                { \node[rsContentBox] at (Aux.center) {\textbullet}; }
                { }
            }
        }
        {   
            \IfBooleanTF { ##1 } { } { \cs:w rs #1 Acr \cs_end: }
            \tl_if_blank:nTF { ##3 } {  }
            {
                \IfBooleanTF{ ##1 } { } {,~}
                ##3            
            }
        }
        
    \group_end:
    }
}


% Команды для сокращений для различных по умолчанию и те, которые
% можно менять
\cs_new_protected:Nn \l_set_pos_acr_code_gen_cs:nnn {
    \tl_const:cn
    { c__rslingu_syntax_ #1 _default_short_ #2 _tl }
    { #3 }  
    
    \tl_set_eq:cc
    { l__rslingu_syntax_ #1 _short_ #2 _tl }
    { c__rslingu_syntax_ #1 _default_short_ #2 _tl }
}


% Существительное


\l_set_pos_acr_code_gen_cs:nnn { noun } { russian } { сущ. } 
\l_set_pos_acr_code_gen_cs:nnn { noun } { ukranian } { ім. }
\l__rslingu_syntax_make_speech_part_cmd_cs:n { Noun }


% Глагол

\l_set_pos_acr_code_gen_cs:nnn { verb } { russian } { глаг. }
\l_set_pos_acr_code_gen_cs:nnn { verb } { ukranian } { дієсл. }
\l__rslingu_syntax_make_speech_part_cmd_cs:n { Verb }


% Прилагательное

\l_set_pos_acr_code_gen_cs:nnn { adjective } { russian } { прил. }
\l_set_pos_acr_code_gen_cs:nnn { adjective } { ukranian } { прикм. }
\l__rslingu_syntax_make_speech_part_cmd_cs:n { Adjective }


% Наречие

\l_set_pos_acr_code_gen_cs:nnn { adverb } { russian } { нареч. }
\l_set_pos_acr_code_gen_cs:nnn { adverb } { ukranian } { присл. }
\l__rslingu_syntax_make_speech_part_cmd_cs:n { Adverb }


% Предлог

\l_set_pos_acr_code_gen_cs:nnn { preposition } { russian } { предлог }
\l_set_pos_acr_code_gen_cs:nnn { preposition } { ukranian } { прийм. }
\l__rslingu_syntax_make_speech_part_cmd_cs:n { Preposition }


% Союз

\l_set_pos_acr_code_gen_cs:nnn { conjunction } { russian } { союз }
\l_set_pos_acr_code_gen_cs:nnn { conjunction } { ukranian } { спол. }
\l__rslingu_syntax_make_speech_part_cmd_cs:n { Conjunction }


% Местоимение

\l_set_pos_acr_code_gen_cs:nnn { pronoun } { russian } { мест. }
\l_set_pos_acr_code_gen_cs:nnn { pronoun } { ukranian } { займ. }
\l__rslingu_syntax_make_speech_part_cmd_cs:n { Pronoun }


% Чатица

\l_set_pos_acr_code_gen_cs:nnn { particle } { russian } { част. }
\l_set_pos_acr_code_gen_cs:nnn { particle } { ukranian } { част. }
\l__rslingu_syntax_make_speech_part_cmd_cs:n { Particle }


% Числительное

\l_set_pos_acr_code_gen_cs:nnn { numeral } { russian } { числ. }
\l_set_pos_acr_code_gen_cs:nnn { numeral } { ukranian } { числ. }
\l__rslingu_syntax_make_speech_part_cmd_cs:n { Numeral }


% Междометие

\l_set_pos_acr_code_gen_cs:nnn { interjection } { russian } { междом. }
\l_set_pos_acr_code_gen_cs:nnn { interjection } { ukranian } { вигук }
\l__rslingu_syntax_make_speech_part_cmd_cs:n { Interjection }


% Причастие

\l_set_pos_acr_code_gen_cs:nnn { participle } { russian } { прич. }
\l_set_pos_acr_code_gen_cs:nnn { participle } { ukranian } { дієприкм. }
\l__rslingu_syntax_make_speech_part_cmd_cs:n { Participle }


% Деепричастие

\l_set_pos_acr_code_gen_cs:nnn { transgressive } { russian } { дееприч. }
\l_set_pos_acr_code_gen_cs:nnn { transgressive } { ukranian } { дієприсл. }
\l__rslingu_syntax_make_speech_part_cmd_cs:n { Transgressive }



