\section{Части речи}\label{sec:speech_parts}

Все частеречные команды имеют одинаковую сигнатуру:
\begin{tcolorbox}
    \rsCode{%
        \rsModifier[cmd]
        {}::
        \rsName[<pos\_cmd\_name>]
        \rsOpt{ \textasteriskcentered{} }
        \rsReq{
            \rsArg[слово:tl]
        }
        \rsOpt{ \rsArg[частеречный\_анализ:tl] }
    }
    \begin{itemize}
        \item \textasteriskcentered{} — спецсимвол печати сокращённого названия части речи над словом.
    \end{itemize}
\end{tcolorbox}

где \rsName[<pos\_cmd\_name>] — название команды для части речи из \autoref{tab:pos-commands}.

\begin{table}[ht!]
    \centering
    \begin{tabular}{@{}llll@{}}
        \toprule
        \rsName[<pos\_cmd\_name>] & Перевод         & \rsName[<pos\_cmd\_name>] & Перевод
        \\\midrule

        \rsName[rsNoun]           & существительное
                                  &
        \rsName[rsVerb]           & глагол
        \\\midrule

        \rsName[rsAdverb]         & наречие
                                  &
        \rsName[rsProposition]    & предлог
        \\\midrule

        \rsName[rsConjunction]    & союз
                                  &
        \rsName[rsPronoun]        & местоимение
        \\\midrule

        \rsName[rsAdjective]      & прилагательное
                                  &
        \rsName[rsParticle]       & частица
        \\\midrule

        \rsName[rsInterjection]   & междометие
                                  &
        \rsName[rsNumeral]        & числительное
        \\\midrule


        \rsName[rsParticiple]     & причастие
                                  &
        \rsName[rsTransgressive]  & деепричастие

        \\\bottomrule
    \end{tabular}
    \caption{Команды для частей речи}
    \label{tab:pos-commands}
\end{table}

\begin{tnote}
    Команды данного раздела чувствительны к опции \rsCode{phantom} внутри окружения
    \macCode{rslingu} или команд для синтаксического разбора предложений.
    Примеры этого можно найти в \autoref{tab:subject-usage} или на
    \autoref{fig:rslingu-demo-full}.
\end{tnote}

Базовые сценарии использования команд для частей речи на примере \rsModifier[rsNoun] представлены в таблице ниже.

\ExplSyntaxOn{}
\begin{table}[ht!]
    \centering
    \begin{tabular}{@{}ll@{}}
        \toprule

        \rsModifier*[rsNoun]
        \rsReq{ \rsArg[существительное] }
         &
        \rsNoun{существительное}
        \\\midrule

        \rsModifier*[rsNoun]
        \rsReq{ \rsArg[существительное] }
        \rsOpt{ \rsArg[ср.р., им.п.] }
         &
        \rsNoun{существительное}[ ср.р., им.п. ]
        \\\midrule

        \rsModifier*[rsNoun]
        \textasteriskcentered{}
        \rsReq{ \rsArg[существительное] }
         &
        \rsNoun*{существительное}
        \\\midrule

        \rsModifier*[rsNoun]
        \textasteriskcentered{}
        \rsReq{ \rsArg[существительное] }
        \rsOpt{ \rsArg[СУЩ.] }
         &
        \rsNoun*{существительное}[СУЩ.]
        \\\midrule

        \rsModifier*[rsNoun]
        \textasteriskcentered{}
        \rsReq{ \rsArg[существительное] }
        \rsOpt{ \rsArg[СУЩ., ср.р., им.п] }
         &
        \rsNoun*{существительное}[СУЩ., ср.р., им.п]
        \\\midrule

        \rsModifier*[rsNoun]
        \rsReq{ \rsArg[существительное] }
        \rsOpt{ \rsArg[СУЩ., ср.р., им.п] }
         &
        \rsNoun{существительное}[СУЩ., ср.р., им.п]
        \\\midrule

        \bottomrule
    \end{tabular}
    \caption{Использование~команд~частей~речи}
\end{table}
\ExplSyntaxOff{}


\subsection{Сокращённые названия частей речи}

Для отображения сокращённых названий частей речи в тексте можно воспользоваться следующей командой
\rsName[rsShowAcr].

\begin{tcolorbox}
    \rsCode{
        \rsModifier[cmd]
        {}::
        \rsName[rsShowAcr]
        \rsOpt{\textasteriskcentered{}}
        \rsOpt{\rsArg[язык:tl] }
        \rsReq{\rsArg[часть\_речи:tl]}
    }
    \textit{
        \begin{itemize}
            \item \textasteriskcentered{} — спецсимвол для печати значения по сокращения умолчнию.
        \end{itemize}
    }
\end{tcolorbox}



\begin{figure}[H]
    \centering
    \begin{minipage}[c]{0.5\textwidth}
        \begin{Latexcode}
            \rsShowAcr{noun} \rsShowAcr*{noun}
            \newline
            \rsSetAcr[russian]{noun}{42}
            \rsShowAcr{noun} \rsShowAcr*{noun}
            \newline
            \rsResetAcr[russian]{noun}
            \rsShowAcr{noun} \rsShowAcr*{noun}
        \end{Latexcode}
    \end{minipage}
    \hfill
    \begin{minipage}[c]{0.4\textwidth}
        \small
        \rsShowAcr{noun} \rsShowAcr*{noun}
        \newline
        \rsSetAcr[russian]{noun}{42}
        \rsShowAcr{noun} \rsShowAcr*{noun}
        \newline
        \rsResetAcr[russian]{noun}
        \rsShowAcr{noun} \rsShowAcr*{noun}
    \end{minipage}

    \caption{Пример печати частечеречных сокращений}
\end{figure}


\subsubsection{Модификация названий сокращений по умолчанию}


\begin{itemize}
    \item
          \rsName[rsSetAcr] позволяет задать частеречное сокращение отличное от значения по умолчанию.
          \begin{tcolorbox}
              \rsModifier[cmd]
              {}::
              \rsName[ rsSetAcr ]
              \rsOpt{ \rsArg[язык:tl] }
              \rsReq{ \rsArg[часть\_речи:tl]  }
              \rsReq{ \rsArg[сокращение:tl]  }
          \end{tcolorbox}

    \item
          \rsName[rsResetAcr] возвращает частеречное сокращение по умолчанию.
          \begin{tcolorbox}
              \rsModifier[cmd]
              {}::
              \rsName[ rsResetAcr ]
              \rsOpt{ \rsArg[язык:tl] }
              \rsReq{ \rsArg[часть\_речи:tl]  }
          \end{tcolorbox}
\end{itemize}


\begin{figure}[H]
    \centering
    \begin{minipage}[c]{0.5\textwidth}
        \begin{Latexcode}
            \rsNoun{существительное}
            \rsSetAcr[russian]{noun}{42}
            \rsNoun{существительное}
            \rsResetAcr[russian]{noun}
            \rsNoun{существительное}
        \end{Latexcode}
    \end{minipage}
    \hfill
    \begin{minipage}[c]{0.4\textwidth}
        \small
        \rsNoun{существительное}
        \rsSetAcr[russian]{noun}{42}
        \rsNoun{существительное}
        \rsResetAcr[russian]{noun}
        \rsNoun{существительное}
    \end{minipage}

    \caption{Пример смены частечеречных сокращений}
\end{figure}


\subsubsection{Локализация стандартных сокращений частеречных команд}

Пакет \rsCode{rslingu} поддерживает локализацию для частеречных сокращений во всех командах из
\namedautoref{tab:pos-commands} через макрос \rsName[rsSetLanguage]:
\ExplSyntaxOn{}
\begin{tcolorbox}
    \rsCode{
        \rsModifier[cmd]
        {}::
        \rsName[rsSetLanguage]
        \rsReq{ \rsArg[язык: tl] }
    }
\end{tcolorbox}
\ExplSyntaxOff{}


Локализация частеречных сокращений доступна для следующих языков:
% template: {{ available_pos_langs }}

Языком по умолчанию является русский. Со списком стандартных сокращений для разных языков
можно ознакомиться в \namedautoref{appendix:pos-contractions}.

\begin{figure}[H]
    \centering
    \begin{minipage}[c]{0.5\textwidth}
        \begin{Latexcode}
            \rsSetLanguage{russian}  % По умолчанию.
            \rsParticiple{Уставшая} \rsNoun{мама}
            \rsVerb{мыла} \rsNoun{раму}
            \rsNoun{вечером}.

            \rsSetLanguage{ukranian}
            \rsVerb{Мріють} \rsNoun{крилами}
            \rsPreposition{з} \rsNoun{туману}
            \rsNoun{лебеді} \rsAdjective{рожеві}.
        \end{Latexcode}
    \end{minipage}
    \hfill
    \begin{minipage}[c]{0.4\textwidth}
        \small
        \rsSetLanguage{russian}  % Язык по умолчанию.
        \rsParticiple{Уставшая}
        \rsNoun{мама} \rsVerb{мыла} \rsNoun{раму} \rsNoun{вечером}.
        \vspace*{\baselineskip}

        \rsSetLanguage{ukranian}
        \rsVerb{Мріють} \rsNoun{крилами} \rsPreposition{з}
        \rsNoun{туману} \rsNoun{лебеді} \rsAdjective{рожеві}.
    \end{minipage}

    \caption{Пример смены языка частеречных сокращений}
\end{figure}