% Кодогенерация команд для изменения цветов.
\cs_new_protected:Nn \g__rslingu_color_codegen_cs:nn {

    \tl_const:cn
        { c__rslingu_ #1 _default_color_tl }
        { #2 }

    \tl_gset_eq:cc
        { g__rslingu_ #1 _color_tl }
        { c__rslingu_ #1 _default_color_tl }

    \definecolor
        { rs #1 default color }
        { HTML }
        { \tl_use:c { c__rslingu_ #1 _default_color_tl } }

    \definecolor
        { rs #1 color }
        { HTML }
        { \tl_use:c { g__rslingu_ #1 _color_tl } }

    \exp_after:wN
    \NewDocumentCommand
    \cs:w
        rsSet
        \text_titlecase_first:n { #1 }
        Color
    \cs_end:
    { m }
    {
        \tl_set:co { g__rslingu_ #1 _color_tl } { ##1 }

        \definecolor
            { rs #1 color }
            { HTML }
            { \tl_use:c { g__rslingu_ #1 _color_tl } }
    }

    \exp_after:wN
    \NewDocumentCommand
    \cs:w
        rsReset
        \text_titlecase_first:n { #1 }
        Color
    \cs_end:
    { }
    {
         \tl_gset_eq:cc
            { g__rslingu_ #1 _color_tl }
            { c__rslingu_ #1 _default_color_tl }

        \definecolor
            { rs #1 color }
            { HTML }
            { \tl_use:c { g__rslingu_ #1 _color_tl } }
    }

    \exp_after:wN
    \NewDocumentCommand
    \cs:w
        rsDefault
        \text_titlecase_first:n { #1 }
        Color
    \cs_end:
    { }
    {
        \tl_use:c { c__rslingu_ #1 _default_color_tl }
    }

    \exp_after:wN
    \NewDocumentCommand
    \cs:w
        rsCurrent
        \text_titlecase_first:n { #1 }
        Color
    \cs_end:
    { }
    {
        \tl_use:c { g__rslingu_ #1 _color_tl }
    }
}

% Кодогенерация приватных команд для морфем.
\cs_new_protected:Nn \g__rslingu_morphology_private_codegen_cs:nn {
    \exp_after:wN
    \cs_new_protected:Nn
    \cs:w
        l__rslingu_morpheme_ #1 _cs:n 
    \cd_end:
    {
        \group_begin:
        \clist_map_variable:nNn { ##1 } { \l__item } {
            \tikz[baseline=(Baseline.base)] {
                \l__rslingu_morpheme_place_content_cs:V \l__item
                \l__rslingu_morpheme_set_color_cs:n { #1 }
                \draw[
                    rsMorphemeSign,
                    \l__rslingu_morpheme_color_tl,
                    transform~canvas={
                        yshift=\c__rslingu_morpheme_sign_y_pad_dim
                    }
                ]
                let
                    \p1 = (Baseline.north~west),
                    \p2 = (Baseline.north~east)
                in
                    % Код для отрисовки.
                    #2
            }
        }
        \group_end:
    }
}

% Кодогенерация публичных команд для морфем.
% #1 — имя морфемы.
\cs_new_protected:Nn \g__rslingu_morphology_public_codegen_cs:n {
    \exp_after:wN
    \NewDocumentCommand
    \cs:w
        rs
        \text_titlecase_first:n { #1 }
    \cs_end:
    { O{} m }
    {
        \keys_set_known:nn { rslingu/morphemes/keys } { phantom=false, color=false, ##1 }

        \tl_if_empty:nTF { ##2 } {
            \bool_if:NT { \l__rslingu_morpheme_phantom_bool } {
                \use:c { l__rslingu_morpheme_ #1 _cs:n } { o }
            }
        } {
            \use:c { l__rslingu_morpheme_ #1 _cs:n } { ##2 }
        }
    }
}

% `textoverset`, аналог `overset`.
\NewDocumentCommand{\textoverset}{ O{.5pt} m m }{
    \tikz[baseline=(Baseline.base)]{
        \node[
            rsContentBox,
            label={
                    [
                            rsContentBox,
                            font=\tiny\itshape,
                            yshift=#1
                        ] #3
                }
        ] (Baseline) { #2 \c__rslingu_above_vmax_phantom_tl };

    }
}

\ExplSyntaxOff

% Правильное подчёркивание-змейка для определений.
% https://tex.stackexchange.com/a/60757/185741
\newif\ifstartcompletesineup
\newif\ifendcompletesineup
\pgfkeys{
    /pgf/decoration/.cd,
    start up/.is if=startcompletesineup,
    start up=true,
    start up/.default=true,
    start down/.style={/pgf/decoration/start up=false},
    end up/.is if=endcompletesineup,
    end up=true,
    end up/.default=true,
    end down/.style={/pgf/decoration/end up=false}
}

\pgfdeclaredecoration{complete sines}{initial}
{
    \state{initial}[
        width=+0pt,
        next state=upsine,
        persistent precomputation={
                \ifstartcompletesineup
                    \pgfkeys{/pgf/decoration automaton/next state=upsine}
                    \ifendcompletesineup
                        \pgfmathsetmacro\matchinglength{
                            0.5*\pgfdecoratedinputsegmentlength / (ceil(0.5* \pgfdecoratedinputsegmentlength / \pgfdecorationsegmentlength) )
                        }
                    \else
                        \pgfmathsetmacro\matchinglength{
                            0.5 * \pgfdecoratedinputsegmentlength / (ceil(0.5 * \pgfdecoratedinputsegmentlength / \pgfdecorationsegmentlength ) - 0.499)
                        }
                    \fi
                \else
                    \pgfkeys{/pgf/decoration automaton/next state=downsine}
                    \ifendcompletesineup
                        \pgfmathsetmacro\matchinglength{
                            0.5* \pgfdecoratedinputsegmentlength / (ceil(0.5 * \pgfdecoratedinputsegmentlength / \pgfdecorationsegmentlength ) - 0.4999)
                        }
                    \else
                        \pgfmathsetmacro\matchinglength{
                            0.5 * \pgfdecoratedinputsegmentlength / (ceil(0.5 * \pgfdecoratedinputsegmentlength / \pgfdecorationsegmentlength ) )
                        }
                    \fi
                \fi
                \setlength{\pgfdecorationsegmentlength}{\matchinglength pt}
            }] {}
    \state{downsine}[width=\pgfdecorationsegmentlength,next state=upsine]{
        \pgfpathsine{\pgfpoint{0.5\pgfdecorationsegmentlength}{0.5\pgfdecorationsegmentamplitude}}
        \pgfpathcosine{\pgfpoint{0.5\pgfdecorationsegmentlength}{-0.5\pgfdecorationsegmentamplitude}}
    }
    \state{upsine}[width=\pgfdecorationsegmentlength,next state=downsine]{
        \pgfpathsine{\pgfpoint{0.5\pgfdecorationsegmentlength}{-0.5\pgfdecorationsegmentamplitude}}
        \pgfpathcosine{\pgfpoint{0.5\pgfdecorationsegmentlength}{0.5\pgfdecorationsegmentamplitude}}
    }
    \state{final}{}
}

\ExplSyntaxOn