\section{Условные обозначения}


\subsection*{Сигнатура макросов}

Сигнатура всех макросов, или текстовых модификаторов, данного пакета задаётся следующим образом.
\begin{tcolorbox}
    \rsCode{%
        \rsModifier{}
        {}::
        \rsName{} \\\rsTab{}
        \rsOpt{
            \rsKwarg{}:\rsType{}=\rsDefault{},
            \rsKwarg{}:\rsType{}=\rsDefault{},
            \ldots,
            \rsKwarg{}:\rsType{}=\rsDefault{}
        } \\\rsTab{}
        \rsReq{ \rsArg{}:\rsType{} }
        \rsOpt{ \rsArg{}:\rsType{} }
        \ldots{}
        \rsReq{ \rsArg{}:\rsType{} } \\\\
    }
    \textit{Опциональная документация макроса.}
\end{tcolorbox}

\begin{itemize}
    \item
          \rsModifier{} — тип макроса. Может быть либо командой (\rsModifier[cmd]),
          либо окружением (\rsModifier[env]).

    \item
          \rsName{} — имя макроса.

    \item
          \rsKwarg{}:\rsType{}=\rsDefault{} — именованный аргумент с именем
          \rsKwarg{} типом \rsType{} и значением по умолчанию \rsDefault{}.

    \item
          \rsArg{}:\rsType{} — позиционный аргумент с именем \rsArg{} и типом \rsType{}.

    \item
          \rsReq{\ldots} / \rsOpt{\ldots} — обозначения для обязательных и необязательных аргументов.
\end{itemize}


\subsection*{Типы аргументов}

\begin{itemize}
    \item
          \rsType[tl] (от англ. \textit{token list}) — произвольный набор символов, обрабатывается целиком.

    \item
          \rsType[clist] (от англ. \textit{comma list}) — набор произвольных символов,
          разделённых запятой, где
          каждый поднабор до запятой обрабатывается независимо.

    \item
          \rsType[bool] — логический тип, может принимать значения «true» или «false».
\end{itemize}
