\section{Прочее}

\subsection{Цвета в морфемном и синтаксических разборах}

Каждая из команд морфемного или синтаксического разбора имеет необязательный
флаг \rsKwarg[color], который отвечает за раскрашивание обозначений морфем
или синтаксических подчёркиваний.

По умолчанию цвета элементов следующие:

% template: {{ morphology_color_table }}

% template: {{ syntax_color_table }}

Ниже представлены макросы для работы с цветом морфем и синтаксических подчёркиваний.
\begin{itemize}
    \item
          \rsName[rsSetColor] позволяет задать цвет элемента отличный от цвета по
          умолчанию.
          \begin{tcolorbox}
              \rsModifier[cmd]
              {}::
              \rsName[rsSetColor]
              \rsReq{\rsArg[имя\_элемента:tl]}
              \rsReq{\rsArg[код\_цвета:tl] }
          \end{tcolorbox}
    \item
          \rsName[rsResetColor] возвращает цвет элемента по умолчанию.
          \begin{tcolorbox}
              \rsModifier[cmd]
              {}::
              \rsName{rsResetColor}
              \rsReq{\rsArg[имя\_элемента:tl]}
          \end{tcolorbox}
\end{itemize}

где \rsArg[имя\_элемента:tl] --- обозначение морфемы или синтаксической конструкции в любом
регистре на английском языке, \rsArg[код\_цвета:tl] --- код цвета в шестнадцатеричном формате.

\begin{figure}[H]
    \centering
    \begin{minipage}[c]{0.5\textwidth}
        \begin{Latexcode}
            \rsRoot[color]{корень}
            \rsSetColor{root}{50c878}
            \rsRoot[color]{корень}
            \rsSetColor{Root}{8f00ff}
            \rsRoot[color]{корень}
            \rsResetColor{root}
            \rsRoot[color]{корень}
        \end{Latexcode}
    \end{minipage}
    \hfill
    \begin{minipage}[c]{0.4\textwidth}
        \small
        \rsRoot[color]{корень}
        \rsSetColor{root}{50c878}
        \rsRoot[color]{корень}
        \rsSetColor{Root}{8f00ff}
        \rsRoot[color]{корень}
        \rsResetColor{root}
        \rsRoot[color]{корень}
    \end{minipage}

    \caption{Пример работы с цветом}
\end{figure}




\subsection{Окружение rslingu}

\ExplSyntaxOn{}
\begin{tcolorbox}
    \rsCode{
        \rsModifier[env]
        {}::
        \rsMan{rslingu} \\\rsTab{}
        \rsOpt{
            \rsKwarg[color]:\rsType[bool]=false,
            \rsKwarg[phantom]:\rsType[bool]=false
        }
    }
\end{tcolorbox}
\ExplSyntaxOff{}

Иногда может возникать необходимость, например, морфемного разбора слов с <<разрывной>> основой. Для таких случаев нет специально
определённых команд, подобно команде \rsCode{rsMorphemicAnalysis}, так что единственный способ отобразить такие слова --- это
последовательное использование команд для каждой из морфем. При передаче параметров \rsCode{phantom} и \rsCode{color} в каждую из
команд возникает многократное дублирование кода, что ухудшает его читаемость.

Решить эту проблему призвано окружение \rsCode{rslingu}, которое указании какого-либо дополнительного аргумента, <<активирует>> его для всех команд, принимающий данный аргумент, внутри окружения.
\begin{figure}[htp!]
    \centering
    \begin{subfigure}{\textwidth}
        \begin{Latexcode}
            \begin{rslingu}[color]
                \rsAttribute{Уставшая} \rsSubject{\rsNoun{мама}} \rsPredicate{\rsVerb*{мыла}}
                \rsObject{раму} \rsAdverbial{вечером}.
            \end{rslingu}
        \end{Latexcode}
        \caption{Код с использованием окружения \rsCode{rslingu}.}
    \end{subfigure}\vspace*{.75cm}
    \begin{subfigure}{\textwidth}
        \begin{Latexcode}
            \rsAttribute[color]{Уставшая} \rsSubject[color]{\rsNoun{мама}}
            \rsPredicate[color]{\rsVerb*{мыла}} \rsObject[color]{раму}
            \rsAdverbial[color]{вечером}.
        \end{Latexcode}
        \caption{Код без использования окружения \rsCode{rslingu}.}
    \end{subfigure}\vspace*{.75cm}
    \begin{subfigure}{.9\textwidth}
        \centering
        \begin{rslingu}[color]
            \rsAttribute{Уставшая} \rsSubject{\rsNoun{мама}} \rsPredicate{\rsVerb*{мыла}}
            \rsObject{раму} \rsAdverbial{вечером}.
        \end{rslingu}
        \caption{Результат выполнения каждого из частей кода выше.}
    \end{subfigure}
    \caption{Демонстрация возможностей окружения \rsCode{rslingu} c параметром \rsCode{color}.}
\end{figure}



\begin{figure}[htp!]
    \centering
    \begin{subfigure}{\textwidth}
        \begin{Latexcode}
            \begin{rslingu}[color,phantom]
                \rsAttribute{Уставшая} \rsSubject{\rsNoun{мама}} \rsPredicate{\rsVerb*{мыла}}
                \rsObject{раму} \rsAdverbial{вечером}.
            \end{rslingu}
        \end{Latexcode}
        \caption{Код с использованием окружения \rsCode{rslingu}.}
    \end{subfigure}\vspace*{.75cm}
    \begin{subfigure}{\textwidth}
        \begin{Latexcode}
            \rsAttribute[color,phantom]{Уставшая} \rsSubject[color,phantom]{\rsNoun{мама}}
            \rsPredicate[color,phantom]{\rsVerb*{мыла}} \rsObject[color,phantom]{раму}
            \rsAdverbial[color,phantom]{вечером}.
        \end{Latexcode}
        \caption{Код без использования окружения \rsCode{rslingu}.}
    \end{subfigure}\vspace*{.75cm}
    \begin{subfigure}{.9\textwidth}
        \centering
        \begin{rslingu}[color,phantom]
            \rsAttribute{Уставшая} \rsSubject{\rsNoun{мама}} \rsPredicate{\rsVerb*{мыла}}
            \rsObject{раму} \rsAdverbial{вечером}.
        \end{rslingu}
        \caption{Результат выполнения каждого из частей кода выше.}
    \end{subfigure}
    \caption{Демонстрация возможностей окружения \rsCode{rslingu} c параметрами \rsCode{color} и \rsCode{phantom}.}
    \label{fig:rslingu-demo-full}
\end{figure}



\clearpage

\begin{appendices}

    \section{Цвета}\label{appendix:colors}

    \section{Сокращения частеречных команд}\label{appendix:pos-contractions}

    \subsection{Русский язык}

    \ExplSyntaxOn{}

    \group_begin:

    \cs_new:Npn \l_tmpa_cs:nn #1#2 {
        \rsName*[ #1 ]\rsReq{\rsArg[ #2 ]} &
        \cs:w #1 \cs_end: { #2 }
    }

    \rsSetLanguage{russian}

    \begin{table}[ht!]
        \footnotesize\centering
        \begin{tabular}{@{}llll@{}}
            \toprule

            \l_tmpa_cs:nn { rsNoun } { существительное }     &
            \l_tmpa_cs:nn { rsVerb } { глагол }
            \\ \midrule
            \l_tmpa_cs:nn { rsAdjective } { прилагательное } &
            \l_tmpa_cs:nn { rsAdverb } { наречие }
            \\ \midrule
            \l_tmpa_cs:nn { rsPronoun } { местоимение }      &
            \l_tmpa_cs:nn { rsPreposition } { предлог }
            \\ \midrule
            \l_tmpa_cs:nn { rsConjunction } { союз }         &
            \l_tmpa_cs:nn { rsParticle } { частица }
            \\ \midrule
            \l_tmpa_cs:nn { rsNumeral } { числительное }     &
            \l_tmpa_cs:nn { rsInterjection } { междометие }
            \\ \midrule
            \l_tmpa_cs:nn { rsParticiple } { причастие }     &
            \l_tmpa_cs:nn { rsTransgressive } { деепричастие }
            \\
            \bottomrule
        \end{tabular}
        \caption{Частеречные~сокращения~для~русского~языка}
    \end{table}

    \group_end:

    \ExplSyntaxOff{}


    \subsection{Украинский язык}

    \ExplSyntaxOn{}

    \group_begin:

    \rsSetLanguage{ukranian}

    \begin{table}[ht!]
        \footnotesize\centering
        \begin{tabular}{@{}llll@{}}
            \toprule

            \l_tmpa_cs:nn { rsNoun } { іменник }             &
            \l_tmpa_cs:nn { rsVerb } { дієслово }
            \\ \midrule
            \l_tmpa_cs:nn { rsAdjective } { прикметник }     &
            \l_tmpa_cs:nn { rsAdverb } { прислівник }
            \\ \midrule
            \l_tmpa_cs:nn { rsPronoun } { займенник }        &
            \l_tmpa_cs:nn { rsPreposition } { прийменник }
            \\ \midrule
            \l_tmpa_cs:nn { rsConjunction } { сполучник }    &
            \l_tmpa_cs:nn { rsParticle } { частка }
            \\ \midrule
            \l_tmpa_cs:nn { rsNumeral } { числівник }        &
            \l_tmpa_cs:nn { rsInterjection } { вигук }
            \\ \midrule
            \l_tmpa_cs:nn { rsParticiple } { дієприкметник } &
            \l_tmpa_cs:nn { rsTransgressive } { дієприслівник }
            \\
            \bottomrule
        \end{tabular}
        \caption{Частеречные~сокращения~для~украинского~языка}
    \end{table}


    \cs_undefine:N \l_tmpa_cs:nn

    \group_end:

    \ExplSyntaxOff{}

\end{appendices}