\section{Синтаксический разбор предложений}\label{sec:SyntaxAnalysis}


Все команды данной группы имеют следующую сигнатуру:
\ExplSyntaxOn
\begin{tcolorbox}
    \manModifier[cmd] \manColon{} \manCode{<name>}
    \manOpt{ \manKwargs{} }
    \manReq{ \manArg[текст:tl] } \\

    \manTab{} \manKwargs{} \manColon{} \manCode{color=false, \manSpace{} phantom=false}
\end{tcolorbox}
\ExplSyntaxOff
где \manCode{<name>} может быть одним из следующих \manModifier[значений]:

\begin{table}[ht!]
    \centering
    \begin{tabular}{@{}llll@{}}
        \toprule

        \manModifier[rsSubject]     & подлежащее
                                   &
        \manModifier[rsPredicate]   & сказуемое
        \\\midrule

        \manModifier[rsObject]      & дополнение
                                   &
        \manModifier[rsAttribute]   & определение
        \\\midrule

        \manModifier[rsAdverbial] & обстоятельство
                                   &
        \\\midrule

        \bottomrule
    \end{tabular}
    \caption{Синтаксические~команды}
\end{table}

\subsection{Подлежащее}


\ExplSyntaxOn

\begingroup
\renewcommand{\arraystretch}{1.125}
\begin{table}[ht!]
    \centering
    \small
    \begin{tabular}{@{}ll@{}}
        \hline

        \manModifier*[rsSubject]
        \manReq{ \manArg[подлежащее] }
         &
        \rsSubject{подлежащее}

        \\\midrule

        \manModifier*[rsSubject]
        \manOpt{ \manKwargs[color] }
        \manReq{ \manArg[подлежащее] }
         &
        \rsSubject[color]{подлежащее}

        \\\midrule

        \manModifier*[rsSubject]
        \manOpt{ \manKwargs[color] }
        \manReq{
            \manModifier*[rsNoun] \manReq{ \manArg[подлежащее] }
            \manSpace{}
            и
            \manSpace{}
            \manModifier*[rsNoun] \manReq{ \manArg[подлежащее] }
        }
         &
        \rsSubject[color]{\rsNoun{подлежащее}[им.п]~и~\rsNoun{подлежащее}[ср.р.]}

        \\\midrule

        \manModifier*[rsSubject]
        \manOpt{ \manKwargs[color,phantom] }
        \manReq{ \manModifier*[rsNoun] \manReq{ \manArg[подлежащее] } }
         &
        \rsSubject[color,phantom]{\rsNoun{подлежащее}}

        \\\midrule

        \manModifier*[rsSubject]
        \manOpt{ \manKwargs[color,phantom] }
        \manReq{ \manArg[подлежащее] }
         &
        \rsSubject[color,phantom]{подлежащее}

        \\\bottomrule
    \end{tabular}
    \caption{Использование~команды~подлежащего}
    \label{tab:subject-usage}
\end{table}
\endgroup


\ExplSyntaxOff


% \subsection{Сказуемое}


% \begingroup
% \renewcommand{\arraystretch}{1.125}
% \begin{table}[ht!]
%     \centering
%     \small
%     \begin{tabular}{|l|l|}
%         \hline
%         \manCode*{rsPredicate{\{сказуемое\}}}                                  & \rsPredicate{сказуемое}                                      \\
%         \manCode*{rsPredicate[color]{\{сказуемое\}}}                           & \rsPredicate[color]{сказуемое}                               \\
%         \manCode*{rsPredicate[phantom, color]{\{сказуемое\}}}                  & \rsPredicate[phantom, color]{сказуемое}                      \\
%         \manCode*{rsPredicate[phantom, color]{\{сказуемое=глаг.\}}}            & \rsPredicate[color]{сказуемое=глаг.}                         \\
%         \manCode*{rsPredicate{\{сказуемое, сказуемое\}}}                       & \rsPredicate{сказуемое, сказуемое}                           \\
%         \manCode*{rsPredicate{\{сказуемое=глаг. + н.в., сказуемое=пр.в.\}}}    & \rsPredicate{сказуемое=глаг. + н.в. + пр.в., сказуемое=б.в.} \\
%         \manCode*{rsPredicate[color, phantom]{\{сказуемое, сказуемое=пр.в.\}}} & \rsPredicate[color, phantom]{сказуемое, сказуемое=пр.в.}     \\
%         \hline
%     \end{tabular}
%     \caption{Использование команды сказуемого.}
% \end{table}
% \endgroup


% \subsection{Дополнение}

% \begingroup
% \renewcommand{\arraystretch}{1.125}
% \begin{table}[ht!]
%     \centering
%     \small
%     \begin{tabular}{|l|l|}
%         \hline
%         \manCode*{rsObject{\{дополнение\}}}                                  & \rsObject{дополнение}                                  \\
%         \manCode*{rsObject[color]{\{дополнение\}}}                           & \rsObject[color]{дополнение}                           \\
%         \manCode*{rsObject[phantom, color]{\{дополнение\}}}                  & \rsObject[phantom, color]{дополнение}                  \\
%         \manCode*{rsObject[phantom, color]{\{дополнение=сущ.\}}}             & \rsObject[color]{дополнение=сущ.}                      \\
%         \manCode*{rsObject{\{дополнение, дополнение\}}}                      & \rsObject{дополнение, дополнение}                      \\
%         \manCode*{rsObject{\{дополнение=сущ., дополнение=сущ. + им.п.\}}}    & \rsObject{дополнение=сущ., дополнение=сущ. + им.п.}    \\
%         \manCode*{rsObject[color, phantom]{\{дополнение, дополнение=сущ.\}}} & \rsObject[color, phantom]{дополнение, дополнение=сущ.} \\
%         \hline
%     \end{tabular}
%     \caption{Использование команды дополнения.}
% \end{table}
% \endgroup



% \subsection{Определение}

% \begingroup
% \renewcommand{\arraystretch}{1.125}
% \begin{table}[ht!]
%     \centering
%     \small
%     \begin{tabular}{|l|l|}
%         \hline
%         \manCode*{rsAttribute{\{определение\}}}                                    & \rsAttribute{определение}                                    \\
%         \manCode*{rsAttribute[color]{\{определение\}}}                             & \rsAttribute[color]{определение}                             \\
%         \manCode*{rsAttribute[phantom, color]{\{определение\}}}                    & \rsAttribute[phantom, color]{определение}                    \\
%         \manCode*{rsAttribute[phantom, color]{\{определение=прил.\}}}              & \rsAttribute[color]{определение=прил.}                       \\
%         \manCode*{rsAttribute{\{определение, определение\}}}                       & \rsAttribute{определение, определение}                       \\
%         \manCode*{rsAttribute{\{определение=прил., определение=прил. + согл.\}}}   & \rsAttribute{определение=прил., определение=прил. + согл.}   \\
%         \manCode*{rsAttribute[color, phantom]{\{определение=прил., определение\}}} & \rsAttribute[color, phantom]{определение=прил., определение} \\
%         \hline
%     \end{tabular}
%     \caption{Использование команды определения.}
% \end{table}
% \endgroup


% \subsection{Обстоятельство}

% \begingroup
% \renewcommand{\arraystretch}{1.125}
% \begin{table}[ht!]
%     \centering
%     \small
%     \begin{tabular}{|l|l|}
%         \hline
%         \manCode*{rsAdverbial{\{обстоятельство\}}}                              & \rsAdverbial{обстоятельство}                     \\
%         \manCode*{rsAdverbial[color]{\{обстоятельство\}}}                       & \rsAdverbial[color]{обстоятельство}              \\
%         \manCode*{rsAdverbial[phantom, color]{\{обстоятельство\}}}              & \rsAdverbial[phantom, color]{обстоятельство}     \\
%         \manCode*{rsAdverbial[phantom, color]{\{обстоятельство=сущ. + им.п.\}}} & \rsAdverbial[color]{обстоятельство=сущ. + им.п.} \\
%         \hline
%     \end{tabular}
%     \caption{Использование команды обстоятельства.}
% \end{table}
% \endgroup
