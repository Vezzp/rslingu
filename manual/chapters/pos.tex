\section{Части речи}\label{sec:speech_parts}

Все частеречные команды имеют одинаковую сигнатуру:
\ExplSyntaxOn
\begin{signature}
    \manModifier[cmd] \manColon{} \manCode{<pos_type>}
    \textasteriskcentered
    \manReq{ \manArg[слово:tl] }
    \manOpt{ \manArg[анализ:tl] }
\end{signature}
\ExplSyntaxOff

где \manCode{<pos\_type>} --- название команды для части речи из \autoref{tab:pos-commands};
\textasteriskcentered --- флаг того, печатать ли сокращённое название части речи над
словом или нет; \manArg[слово:tl] --- слово для частеречного разбора;
\manArg[анализ:tl] --- частеречный анализ слова.

\begin{table}[ht!]
    \centering
    \begin{tabular}{@{}llll@{}}
        \toprule
        \manCode{pos\_type}           & Перевод         & \manCode{pos\_type} & Перевод
        \\\midrule

        \manModifier[rsNoun]          & существительное
                                      &
        \manModifier[rsVerb]          & глагол
        \\\midrule

        \manModifier[rsAdverb]        & наречие
                                      &
        \manModifier[rsProposition]   & предлог
        \\\midrule

        \manModifier[rsConjunction]   & союз
                                      &
        \manModifier[rsPronoun]       & местоимение
        \\\midrule

        \manModifier[rsAdjective]     & прилагательное
                                      &
        \manModifier[rsParticle]      & частица
        \\\midrule

        \manModifier[rsInterjection]  & междометие
                                      &
        \manModifier[rsNumeral]       & числительное
        \\\midrule


        \manModifier[rsParticiple]    & причастие
                                      &
        \manModifier[rsTransgressive] & деепричастие

        \\\bottomrule
    \end{tabular}
    \caption{Команды для частей речи}
    \label{tab:pos-commands}
\end{table}

\begin{tnote}
    Команды данного раздела чувствительны к опции \manCode{phantom} внутри окружения
    \macCode{rslingu} или команд для синтаксического разбора предложений.
    Примеры этого можно найти в \autoref{tab:subject-usage} или на
    \autoref{fig:rslingu-demo-full}.
\end{tnote}

Базовые сценарии использования команд для частей речи на примере \manModifier[rsNoun] представлены в таблице ниже.

\ExplSyntaxOn
\begin{table}[ht!]
    \centering
    \begin{tabular}{@{}ll@{}}
        \toprule

        \manModifier*[rsNoun]
        \manReq{ \manArg[существительное] }
         &
        \rsNoun{существительное}
        \\\midrule

        \manModifier*[rsNoun]
        \manReq{ \manArg[существительное] }
        \manOpt{ \manArg[ср.р., им.п.] }
         &
        \rsNoun{существительное}[ ср.р., им.п. ]
        \\\midrule

        \manModifier*[rsNoun]
        \textasteriskcentered{}
        \manReq{ \manArg[существительное] }
         &
        \rsNoun*{существительное}
        \\\midrule

        \manModifier*[rsNoun]
        \textasteriskcentered{}
        \manReq{ \manArg[существительное] }
        \manOpt{ \manArg[СУЩ.] }
         &
        \rsNoun*{существительное}[СУЩ.]
        \\\midrule

        \manModifier*[rsNoun]
        \textasteriskcentered{}
        \manReq{ \manArg[существительное] }
        \manOpt{ \manArg[СУЩ., ср.р., им.п] }
         &
        \rsNoun*{существительное}[СУЩ., ср.р., им.п]
        \\\midrule

        \manModifier*[rsNoun]
        \manReq{ \manArg[существительное] }
        \manOpt{ \manArg[СУЩ., ср.р., им.п] }
         &
        \rsNoun{существительное}[СУЩ., ср.р., им.п]
        \\\midrule

        \bottomrule
    \end{tabular}
    \caption{Использование~команд~частей~речи}
\end{table}
\ExplSyntaxOff


\subsection{Сокращённые названия частей речи}

Для отображения сокращённых названий частей речи в тексте можно воспользоваться следующей командой:
\ExplSyntaxOn
\begin{signature}
    \manModifier[cmd] \manColon{} \manCode{rsShowAcr}
    \textasteriskcentered
    \manOpt{ \manArg[язык:tl] }
    \manReq{ \manArg[часть_речи:tl]  }
\end{signature}
\ExplSyntaxOff


\begin{figure}[H]
    \centering
    \begin{minipage}[c]{0.5\textwidth}
        \begin{Latexcode}
            \rsShowAcr{noun} \rsShowAcr*{noun}
            \newline
            \rsSetAcr[russian]{noun}{42}
            \rsShowAcr{noun} \rsShowAcr*{noun}
            \newline
            \rsResetAcr[russian]{noun}
            \rsShowAcr{noun} \rsShowAcr*{noun}
        \end{Latexcode}
    \end{minipage}
    \hfill
    \begin{minipage}[c]{0.4\textwidth}
        \small
        \rsShowAcr{noun} \rsShowAcr*{noun}
        \newline
        \rsSetAcr[russian]{noun}{42}
        \rsShowAcr{noun} \rsShowAcr*{noun}
        \newline
        \rsResetAcr[russian]{noun}
        \rsShowAcr{noun} \rsShowAcr*{noun}
    \end{minipage}

    \caption{Пример печати частечеречных сокращений}
\end{figure}


\subsubsection{Модификация названий сокращений по умолчанию}


\begin{itemize}
    \item \manModifier[rsSetAcr] позволяет задать частеречное сокращение отличное от значения по умолчанию.
          \ExplSyntaxOn
          \begin{signature}
              \manModifier[cmd] \manColon{}
              \manCode{ rsSetAcr }
              \manOpt{ \manArg[язык:tl] }
              \manReq{ \manArg[часть_речи:tl]  }
              \manReq{ \manArg[сокращение:tl]  }
          \end{signature}
          \ExplSyntaxOff
    \item \manModifier[rsResetAcr] возвращает частеречное сокращение по умолчанию.
          \ExplSyntaxOn
          \begin{signature}
              \manModifier[cmd] \manColon{}
              \manCode{ rsResetAcr }
              \manOpt{ \manArg[язык:tl] }
              \manReq{ \manArg[часть_речи:tl]  }
          \end{signature}
          \ExplSyntaxOff
\end{itemize}


\begin{figure}[H]
    \centering
    \begin{minipage}[c]{0.5\textwidth}
        \begin{Latexcode}
            \rsNoun{существительное}
            \rsSetAcr[russian]{noun}{42}
            \rsNoun{существительное}
            \rsResetAcr[russian]{noun}
            \rsNoun{существительное}
        \end{Latexcode}
    \end{minipage}
    \hfill
    \begin{minipage}[c]{0.4\textwidth}
        \small
        \rsNoun{существительное}
        \rsSetAcr[russian]{noun}{42}
        \rsNoun{существительное}
        \rsResetAcr[russian]{noun}
        \rsNoun{существительное}
    \end{minipage}

    \caption{Пример смены частечеречных сокращений}
\end{figure}


\subsubsection{Локализация стандартных сокращений частеречных команд}

Пакет \manCode{rslingu} поддерживает локализацию для частеречных сокращений во всех командах из
\namedautoref{tab:pos-commands} через модификатор \manModifier[rsSetLanguage]:
\ExplSyntaxOn
\begin{signature}
    \manCode{
        \manModifier[cmd] \manColon rsSetLanguage
        \manReq{ \manArg[язык: tl] }
    }
\end{signature}
\ExplSyntaxOff


Локализация частеречных сокращений доступна для следующих языков:

        \begin{table}[ht!]
            \centering
            \begin{tabular}{@{}ll@{}}
                \toprule
                Язык & \manArg[язык] \\\midrule
                русский & russian \\\midrule
украинский & ukranian \\\midrule
                \bottomrule
            \end{tabular}
            \caption{Доступные языки для локализации частеречных команд}
        \end{table}


Языком по умолчанию является русский. Со списком стандартных сокращений для разных языков
можно ознакомиться в \namedautoref{appendix:pos-contractions}.

\begin{figure}[H]
    \centering
    \begin{minipage}[c]{0.5\textwidth}
        \begin{Latexcode}
            \rsSetLanguage{russian}  % По умолчанию.
            \rsParticiple{Уставшая} \rsNoun{мама}
            \rsVerb{мыла} \rsNoun{раму}
            \rsNoun{вечером}.

            \rsSetLanguage{ukranian}
            \rsVerb{Мріють} \rsNoun{крилами}
            \rsPreposition{з} \rsNoun{туману}
            \rsNoun{лебеді} \rsAdjective{рожеві}.
        \end{Latexcode}
    \end{minipage}
    \hfill
    \begin{minipage}[c]{0.4\textwidth}
        \small
        \rsSetLanguage{russian}  % Язык по умолчанию.
        \rsParticiple{Уставшая}
        \rsNoun{мама} \rsVerb{мыла} \rsNoun{раму} \rsNoun{вечером}.
        \vspace*{\baselineskip}

        \rsSetLanguage{ukranian}
        \rsVerb{Мріють} \rsNoun{крилами} \rsPreposition{з}
        \rsNoun{туману} \rsNoun{лебеді} \rsAdjective{рожеві}.
    \end{minipage}

    \caption{Пример смены языка частеречных сокращений}
\end{figure}