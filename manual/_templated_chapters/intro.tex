\section{Условные обозначения}

Все макросы, или текстовые модификаторы текста, пакета \texttt{rslingu} в данном руководстве
задаются следующим образом:
\ExplSyntaxOn
\begin{signature}
    \manCode {
        \manModifier[mtype]
        \manSpace{}\manColon{}\manSpace{}
        mname
        \manSpace{}
        \manOpt\manKwargs
        \manSpace{}
        < [, \{ >
                \manSpace{}
                \manArg{}:type_hint
                \manSpace{}< ], \} >
        \manSpace{}... \\

        \manTab{} \manKwargs<> \manSpace{}
        kwarg_name \manSpace{} \manColon{} \manSpace{}type_hint = kwarg_default_value \manSpace{} ...
    }
\end{signature}
\ExplSyntaxOff

\begin{itemize}
    \item \manModifier[modifier\_type] — тип текстового модификатора, команда (\manModifier[cmd]) или окружение (\manModifier[env]).

    \item \manKwargs{} — ненулевое количество именованных необязательных аргументов.

    \item
          \manArg{} — ненулевое количество обязательных и/или необязательных аргументов.
\end{itemize}

По возможности, у каждого агрумента будут указаны подсказки об обозначении аргумента и
его типе (\texttt{type\_hint}), который может принимать следующие значения:

\begin{itemize}
    \item \texttt{tl}, от англ. \textit{token list}, — произвольный набор символов, обрабатывается целиком.

    \item \texttt{clist}, от англ. \textit{comma list}, — набор произвольных символов, разделённых запятой, где
          каждый поднабор до запятой обрабатывается независимо.

    \item \texttt{bool} — логический тип, может принимать значения «true» или «false».
\end{itemize}
