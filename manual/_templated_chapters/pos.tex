\section{Части речи}\label{sec:speech_parts}


\subsection{Базовое использование}

Все команды данной группы имеют следующую сигнатуру:
\ExplSyntaxOn
\begin{tcolorbox}
    \manModifier[cmd] \manColon{} \manCode{<name>}
    \manKwargs[\textasteriskcentered]
    \manReq{ \manArg[<слово:tl>] }
    \manOpt{ \manArg[<анализ:tl>] }
\end{tcolorbox}
\ExplSyntaxOff
где \manCode{<name>} может быть одним из следующих \manModifier[значений]:

\begin{table}[ht!]
    \centering
    \begin{tabular}{@{}llll@{}}
        \toprule

        \manModifier[rsNoun]          & существительное
                                     &
        \manModifier[rsVerb]          & глагол
        \\\midrule

        \manModifier[rsAdverb]        & наречие
                                     &
        \manModifier[rsProposition]   & предлог
        \\\midrule

        \manModifier[rsConjunction]   & союз
                                     &
        \manModifier[rsPronoun]       & местоимение
        \\\midrule

        \manModifier[rsAdjective]     & прилагательное
                                     &
        \manModifier[rsParticle]      & частица
        \\\midrule

        \manModifier[rsInterjection]  & междометие
                                     &
        \manModifier[rsNumeral]       & числительное
        \\\midrule


        \manModifier[rsParticiple]    & причастие
                                     &
        \manModifier[rsTransgressive] & деепричастие

        \\\bottomrule
    \end{tabular}
    \caption{Команды для частей речи}
    \label{tab:pos-commands}
\end{table}

\begin{tnote}
    Команды данного раздела чувствительны к опции \manCode{phantom}.
\end{tnote}

Базовые сценарии использования команд для частей речи на примере \manModifier[rsNoun] представлены в таблице ниже.

\ExplSyntaxOn
\begin{table}[ht!]
    \centering
    \begin{tabular}{@{}ll@{}}
        \toprule

        \manModifier*[rsNoun]
        \manReq{ \manArg[существительное] }
         &
        \rsNoun{существительное}
        \\\midrule

        \manModifier*[rsNoun]
        \manReq{ \manArg[существительное] }
        \manOpt{ \manArg[ср.р., им.п.] }
         &
        \rsNoun{существительное}[ ср.р., им.п. ]
        \\\midrule

        \manModifier*[rsNoun]
        \textasteriskcentered{}
        \manReq{ \manArg[существительное] }
         &
        \rsNoun*{существительное}
        \\\midrule

        \manModifier*[rsNoun]
        \textasteriskcentered{}
        \manReq{ \manArg[существительное] }
        \manOpt{ \manArg[СУЩ.] }
         &
        \rsNoun*{существительное}[СУЩ.]
        \\\midrule

        \manModifier*[rsNoun]
        \textasteriskcentered{}
        \manReq{ \manArg[существительное] }
        \manOpt{ \manArg[СУЩ., ср.р., им.п] }
         &
        \rsNoun*{существительное}[СУЩ., ср.р., им.п]
        \\\midrule
        \bottomrule
    \end{tabular}
    \caption{Использование~команд~частей~речи}
\end{table}
\ExplSyntaxOff


\subsection{Сокращённые названия частей речи}


\subsubsection{Базовое использование}\label{subsubsec:pos-basic}

Для отображения сокращённых названий частей речи в тексте предназначены команды со следующей сигнатурой:
\ExplSyntaxOn
\begin{tcolorbox}
    \manModifier[cmd] \manColon{} \manCode{<name> Acr}
\end{tcolorbox}
\ExplSyntaxOff
где \manCode{<name>} принимает значения команд из таблицы «\nameref{tab:pos-commands}».

Например, \manCode{\manModifier*[rsNounAcr]\manReq{}} даст «\rsNounAcr{}», использование фигурных скобок необходимо для установки пробела после сокращения.


\subsubsection{Локализация стандартных сокращений частеречных команд}

Пакет \manCode{rslingu} поддерживает локализацию для сокращений частей речи во всех командах из
таблицы «\nameref{tab:pos-commands}» через модификатор \manModifier[rsSetLanguage]:
\ExplSyntaxOn
\begin{tcolorbox}
    \manCode{
        \manModifier[cmd] \manColon rsSetLanguage
        \manReq{ \manArg[<язык: tl>] }
    }
\end{tcolorbox}
\ExplSyntaxOff

На данный момент доступны следующие языки:
\begin{table}[ht!]
    \centering
    \begin{tabular}{@{}ll@{}}
        \toprule

        Язык       & \manArg[язык] \\
        \midrule

        русский    & russian      \\
        украинский & ukranian     \\

        \bottomrule
    \end{tabular}
    \caption{Доступные языки для локализации частеречных команд}
\end{table}

Языком по умолчанию является русский. Со списком стандартных сокращений для разных языков
можно ознакомиться в приложении \ref{appendix:pos-contractions}
«\nameref{appendix:pos-contractions}».


\begin{figure}[H]
    \centering
    \begin{minipage}[c]{0.5\textwidth}
        \begin{Latexcode}
            \rsSetLanguage{russian}  % По умолчанию.
            \rsParticiple{Уставшая} \rsNoun{мама}
            \rsVerb{мыла} \rsNoun{раму}
            \rsNoun{вечером}.

            \rsSetLanguage{ukranian}
            \rsVerb{Мріють} \rsNoun{крилами}
            \rsPreposition{з} \rsNoun{туману}
            \rsNoun{лебеді} \rsAdjective{рожеві}.
        \end{Latexcode}
    \end{minipage}
    \hfill
    \begin{minipage}[c]{0.4\textwidth}
        \small
        \rsSetLanguage{russian}  % Язык по умолчанию.
        \rsParticiple{Уставшая}
        \rsNoun{мама} \rsVerb{мыла} \rsNoun{раму} \rsNoun{вечером}.
        \vspace*{\baselineskip}

        \rsSetLanguage{ukranian}
        \rsVerb{Мріють} \rsNoun{крилами} \rsPreposition{з}
        \rsNoun{туману} \rsNoun{лебеді} \rsAdjective{рожеві}.
    \end{minipage}

    \caption{Пример смены языка частеречных сокращений}
\end{figure}