\def\RSUlineYShift{2}
\def\RSUlineWidth{1.25}

\tikzset{
    RSUline/.style={
        line width=\RSUlineWidth pt,
    },
}


\definecolor{RSSubjectColor}{HTML}{673ab7}
\definecolor{RSPredicateColor}{HTML}{e81e62}
\definecolor{RSAttributeColor}{HTML}{2196f3}
\definecolor{RSAdverbarialColor}{HTML}{009688}
\definecolor{RSObjectColor}{HTML}{ffa500}


\ExplSyntaxOn
    \cs_generate_variant:Nn \seq_set_split:Nnn { NnV }

    \cs_new_protected:Nn \l__rs_syntax_choose_text:n {
        \group_begin:
            \def\l__do_spacing{0}
            \node[RSCage] (Root) {
                \clist_map_variable:nNn { #1 } {\l__item} {
                    \FPifeq{\l__do_spacing}{0}
                        \def\l__do_spacing{1}
                    \else
                        \space
                    \fi
                    \StrCut{\l__item}{=}\l__word\l__text
                    \seq_set_split:NnV \l__tmp_seq { + } { \l__text }
                    \bool_if:nTF { \RSPhantomBool || \l_rs_syntax_phantom_bool }
                    {
                        \textoverset{%
                            \tikz[baseline=(tmp.base)]{
                                \node[RSCage] (tmp) {\phantom{\l__word}};
                                \node[RSCage] at (tmp.center) {\textbullet};
                            }
                        }{ \seq_use:Nn \l__tmp_seq {,\space} }
                    }{
                        \textoverset{\l__word}{ \seq_use:Nn \l__tmp_seq {,\space} }
                    }
                    
                }\RSDownVPhantom
            };
        \group_end:
    }
    
    \cs_generate_variant:Nn \l__rs_syntax_choose_text:n { p }

\ExplSyntaxOff




% ==============================================
% =========== Subject (подлежащее) =============
% ==============================================

\ExplSyntaxOn

\NewDocumentCommand{\rsSubject}{ O{} m }{%
    \group_begin:
        \keys_set_known:nn { rs/syntax/keys } {color=false, phantom=false, width, #1}
    
        \tikz[baseline=(Root.base)]{
            \l__rs_syntax_choose_text:p  { #2 }
            \l__rs_choose_color:n { RSSubjectColor }
            \draw[\l__rs_color, RSUline]
                ([yshift=-\RSUlineYShift pt]Root.south~west) --
                ([yshift=-\RSUlineYShift pt]Root.south~east);
        }
    \group_end:
}

\ExplSyntaxOff




% ==============================================
% =========== Predicate (сказуемое) ============
% ==============================================

\ExplSyntaxOn

\NewDocumentCommand{\rsPredicate}{ O{} m }{%
    \group_begin:
        \keys_set_known:nn { rs/syntax/keys } {color=false, phantom=false, width, #1}
    
        \tikz[baseline=(Root.base)]{
            \l__rs_syntax_choose_text:p  { #2 }
            \l__rs_choose_color:n { RSPredicateColor }
            \draw[\l__rs_color, RSUline]
                ([yshift=-\RSUlineYShift pt]Root.south~west) --
                ([yshift=-\RSUlineYShift pt]Root.south~east);
            \draw[\l__rs_color, RSUline]
                ([yshift=-\RSUlineYShift - 2 pt]Root.south~west) --
                ([yshift=-\RSUlineYShift - 2 pt]Root.south~east);
        }
    \group_end:
}

\ExplSyntaxOff





% % ==============================================
% % ======== Adverbarial (обстоятельство) ========
% % ==============================================

\ExplSyntaxOn

\NewDocumentCommand{\rsAdverbarial}{ O{} m }{%
    \group_begin:
        \keys_set_known:nn { rs/syntax/keys } {color=false, phantom=false, width, #1}
    
        \tikz[baseline=(Root.base)]{
            \l__rs_syntax_choose_text:p  { #2 }
            \l__rs_choose_color:n { RSAdverbarialColor }
            \draw[
                \l__rs_color,
                RSUline,
                dash~pattern={on 5pt off 2pt on 1.5pt off 2pt},
            ]
                ([yshift=-\RSUlineYShift pt]Root.south~west) --
                ([yshift=-\RSUlineYShift pt]Root.south~east);
        }
    \group_end:
}

\ExplSyntaxOff





% ==============================================
% ======== Attribute (определение) =============
% ==============================================

\ExplSyntaxOn

\NewDocumentCommand{\rsAttribute}{ O{} m }{%
    \group_begin:
        \keys_set_known:nn { rs/syntax/keys } {color=false, phantom=false, width, #1}
    
        \tikz[baseline=(Root.base)]{
            \l__rs_syntax_choose_text:p  { #2 }
            \l__rs_choose_color:n { RSAttributeColor }
            \draw[
                \l__rs_color,
                RSUline, 
                decorate,
                decoration={
                    complete~sines,
                    segment~length=2.75pt,
                    amplitude=1.75,
                    mirror,
                    start~up,
                    end~down
                }
            ]
                ([yshift=-\RSUlineYShift pt]Root.south~west) --
                ([yshift=-\RSUlineYShift pt]Root.south~east);
        }
    \group_end:
}

\ExplSyntaxOff




% ==============================================
% ======== Adverbarial (обстоятельство) ========
% ==============================================

\ExplSyntaxOn

\NewDocumentCommand{\rsObject}{ O{} m }{%
    \group_begin:
        \keys_set_known:nn { rs/syntax/keys } {color=false, phantom=false, width, #1}
    
        \tikz[baseline=(Root.base)]{
            \l__rs_syntax_choose_text:p  { #2 }
            \l__rs_choose_color:n { RSObjectColor }
            \draw[
                \l__rs_color,
                RSUline, 
                dash~pattern={on 5pt off 2pt on 1pt off 0pt},
            ]
                ([yshift=-\RSUlineYShift pt]Root.south~west) --
                ([yshift=-\RSUlineYShift pt]Root.south~east);
        }
    \group_end:
}

\ExplSyntaxOff




% ==============================================
% =============== Части речи ===================
% ==============================================

\NewDocumentCommand{\rsNoun}{ m O{} }{%
    \textoverset{#1}{сущ.\ifblank{#2}{}{, #2}}%
}


\NewDocumentCommand{\rsVerb}{ m O{} }{%
    \textoverset{#1}{глаг.\ifblank{#2}{}{, #2}}%
}


\NewDocumentCommand{\rsAdverb}{ m O{} }{%
    \textoverset{#1}{нареч.\ifblank{#2}{}{, #2}}%
}


\NewDocumentCommand{\rsPretext}{ m O{} }{%
    \textoverset{#1}{предлог\ifblank{#2}{}{, #2}}%
}


\NewDocumentCommand{\rsUnion}{ m O{} }{%
    \textoverset{#1}{союз\ifblank{#2}{}{, #2}}%
}

\NewDocumentCommand{\rsPronoun}{ m O{} }{%
    \textoverset{#1}{мест.\ifblank{#2}{}{, #2}}%
}

\NewDocumentCommand{\rsAdjective}{ m O{} }{%
    \textoverset{#1}{прил.\ifblank{#2}{}{, #2}}%
}

\NewDocumentCommand{\rsParticle}{ m O{} }{%
    \textoverset{#1}{част.\ifblank{#2}{}{, #2}}%
}