% Цвета синтаксических подчёркиваний.
\definecolor{rsSubjectColor}{HTML}{673ab7}
\definecolor{rsPredicateColor}{HTML}{e81e62}
\definecolor{rsAttributeColor}{HTML}{2196f3}
\definecolor{rsAdverbarialColor}{HTML}{009688}
\definecolor{rsObjectColor}{HTML}{ffa500}


\ExplSyntaxOn
% Цвет синтаксических подчёркиваний.
    \tl_new:N \l__rslingu_syntax_color_tl
    \cs_new_protected:Nn \l__rslingu_syntax_set_color_cs:n {
        \tl_set:Nn \l__rslingu_syntax_color_tl {
            \bool_if:nTF
                { \l__rslingu_syntax_color_bool || \l__rslingu_color_bool}
                { #1 } 
                { black }
        }
    }

    \dim_const:Nn \c__rslingu_syntax_uline_shift_dim { 2pt }
    \dim_const:Nn \c__rslingu_syntax_uline_width_dim { 1.25pt }

    \tikzset{
        rsUline/.style={
            line~width=\c__rslingu_syntax_uline_width_dim,
        },
    }
\ExplSyntaxOff


\ExplSyntaxOn
\cs_generate_variant:Nn \seq_set_split:Nnn { NnV }

% Флаг для постановки пробела между однородными членами, если они есть.
\bool_new:N \l__place_spacing_bool
\cs_new_protected:Nn \l__rslingu_syntax_place_content_cs:n {
    \group_begin:

    \node[rsContentBox] (Baseline) {

        % Разделяем строку по запятым.
        \clist_map_variable:nNn { #1 } \l__homogeneous_item_tl {
            % Проверяем, нужно ли ставить пробел между однородными членами.
            \bool_if:nTF \l__place_spacing_bool {
                {~}
            } {
                \bool_set_true:N \l__place_spacing_bool
            }

            % Разделяем по знаку равно.
            \StrCut{ \l__homogeneous_item_tl } { = } \l__word_tl \l__aux_info_tl

            % Разделяем по знаку плюса.
            \seq_set_split:NnV \l__aux_info_seq { + } { \l__aux_info_tl }

            \tikz[baseline=(Aux.base)] {
                \node[
                    rsContentBox,
                    label={
                        [
                            inner~sep=0pt,
                            outer~sep=0pt,
                            font=\tiny\itshape,
                            yshift=.5pt
                        ]
                        % Печать дополнительной информации, содержащейся в `l__aux_info_seq`,
                        % если та непуста.
                        \seq_if_empty:NTF \l__aux_info_seq { } {
                            \seq_use:Nn \l__aux_info_seq {,~}
                        }
                    }
                ] (Aux) {
                    \bool_if:nTF { \l__rslingu_syntax_phantom_bool || \l__rslingu_phantom_bool } {
                        \phantom{\l__word_tl}
                    } {
                        \l__word_tl
                    }
                    \c__rslingu_above_vmax_phantom_tl
                };

                % NOTE: Дважды проверка одного и того же, пока не знаю, как сделать лучше.
                \bool_if:nTF { \l__rslingu_syntax_phantom_bool || \l__rslingu_phantom_bool } {
                    \node[rsContentBox] at (Aux.center) {\textbullet};
                } { }
            }
        }
        % Выравнивание по высоте внутри блока `Baseline`.
        \c__rslingu_below_vmax_phantom_tl
    };

    \bool_set_false:N \l__place_spacing_bool
    \group_end:
}

\ExplSyntaxOff




% ==============================================
% =========== Subject (подлежащее) =============
% ==============================================

\ExplSyntaxOn

\NewDocumentCommand{\rsSubject} { O{} m } {
    \group_begin:
        \keys_set_known:nn { rslingu/syntax/keys } { color=false, phantom=false, #1 }
        \tikz[baseline=(Baseline.base)] {
            \l__rslingu_syntax_place_content_cs:n { #2 }
            \l__rslingu_syntax_set_color_cs:n { rsSubjectColor }
            \draw[
                rsUline,
                \l__rslingu_syntax_color_tl
            ]
                ([yshift=-\c__rslingu_syntax_uline_shift_dim]Baseline.south~west) --
                ([yshift=-\c__rslingu_syntax_uline_shift_dim]Baseline.south~east);
        }
    \group_end:
}

\ExplSyntaxOff




% % ==============================================
% % =========== Predicate (сказуемое) ============
% % ==============================================

\ExplSyntaxOn

\NewDocumentCommand{\rsPredicate} { O{} m } {
    \group_begin:
        \keys_set_known:nn { rslingu/syntax/keys } { color=false, phantom=false, #1 }
        \tikz[baseline=(Baseline.base)] {
            \l__rslingu_syntax_place_content_cs:n { #2 }
            \l__rslingu_syntax_set_color_cs:n { rsPredicateColor }
            \draw[
                rsUline,
                \l__rslingu_syntax_color_tl
            ]
                ([yshift=-\c__rslingu_syntax_uline_shift_dim]Baseline.south~west) --
                ([yshift=-\c__rslingu_syntax_uline_shift_dim]Baseline.south~east);
            \draw[
                rsUline,
                \l__rslingu_syntax_color_tl
            ]
                ([yshift=-\c__rslingu_syntax_uline_shift_dim - 2pt]Baseline.south~west) --
                ([yshift=-\c__rslingu_syntax_uline_shift_dim - 2pt]Baseline.south~east);
        }
    \group_end:
}

\ExplSyntaxOff





% % ==============================================
% % ======== Adverbarial (обстоятельство) ========
% % ==============================================

\ExplSyntaxOn

\NewDocumentCommand{\rsAdverbarial} { O{} m } {
    \group_begin:
        \keys_set_known:nn { rslingu/syntax/keys } { color=false, phantom=false, #1 }
        \tikz[baseline=(Baseline.base)] {
            \l__rslingu_syntax_place_content_cs:n { #2 }
            \l__rslingu_syntax_set_color_cs:n { rsAdverbarialColor }
            \draw[
                rsUline,
                \l__rslingu_syntax_color_tl,
                dash~pattern={on 5pt off 2pt on 1.5pt off 2pt},
            ]
                ([yshift=-\c__rslingu_syntax_uline_shift_dim]Baseline.south~west) --
                ([yshift=-\c__rslingu_syntax_uline_shift_dim]Baseline.south~east);
        }
    \group_end:
}

\ExplSyntaxOff




% % ==============================================
% % ======== Attribute (определение) =============
% % ==============================================

\ExplSyntaxOn

\NewDocumentCommand{\rsAttribute} { O{} m } {
    \group_begin:
        \keys_set_known:nn { rslingu/syntax/keys } { color=false, phantom=false, #1 }
        \tikz[baseline=(Baseline.base)] {
            \l__rslingu_syntax_place_content_cs:n { #2 }
            \l__rslingu_syntax_set_color_cs:n { rsAttributeColor }
            \draw[
                rsUline,
                \l__rslingu_syntax_color_tl,
                decorate,
                decoration={
                    complete~sines,
                    segment~length=2.75pt,
                    amplitude=1.75,
                    mirror,
                    start~up,
                    end~down
                }
            ]
                ([yshift=-\c__rslingu_syntax_uline_shift_dim]Baseline.south~west) --
                ([yshift=-\c__rslingu_syntax_uline_shift_dim]Baseline.south~east);
        }
    \group_end:
}

\ExplSyntaxOff




% % ==============================================
% % ========== Object (обстоятельство) ===========
% % ==============================================

\ExplSyntaxOn

\NewDocumentCommand{\rsObject} { O{} m } {
    \group_begin:
        \keys_set_known:nn { rslingu/syntax/keys } { color=false, phantom=false, #1 }
        \tikz[baseline=(Baseline.base)] {
            \l__rslingu_syntax_place_content_cs:n { #2 }
            \l__rslingu_syntax_set_color_cs:n { rsObjectColor }
            \draw[
                rsUline,
                \l__rslingu_syntax_color_tl,
                dash~pattern={on 5pt off 2pt on 1pt off 0pt},
            ]
                ([yshift=-\c__rslingu_syntax_uline_shift_dim]Baseline.south~west) --
                ([yshift=-\c__rslingu_syntax_uline_shift_dim]Baseline.south~east);
        }
    \group_end:
}

\ExplSyntaxOff




% % ==============================================
% % =============== Части речи ===================
% % ==============================================

\NewDocumentCommand{\rsNoun}{ m O{} }{%
    \textoverset{#1}{сущ.\ifblank{#2}{}{, #2}}%
}

\NewDocumentCommand{\rsVerb}{ m O{} }{%
    \textoverset{#1}{глаг.\ifblank{#2}{}{, #2}}%
}

\NewDocumentCommand{\rsAdverb}{ m O{} }{%
    \textoverset{#1}{нареч.\ifblank{#2}{}{, #2}}%
}

\NewDocumentCommand{\rsPretext}{ m O{} }{%
    \textoverset{#1}{предлог\ifblank{#2}{}{, #2}}%
}

\NewDocumentCommand{\rsUnion}{ m O{} }{%
    \textoverset{#1}{союз\ifblank{#2}{}{, #2}}%
}

\NewDocumentCommand{\rsPronoun}{ m O{} }{%
    \textoverset{#1}{мест.\ifblank{#2}{}{, #2}}%
}

\NewDocumentCommand{\rsAdjective}{ m O{} }{%
    \textoverset{#1}{прил.\ifblank{#2}{}{, #2}}%
}

\NewDocumentCommand{\rsParticle}{ m O{} }{%
    \textoverset{#1}{част.\ifblank{#2}{}{, #2}}%
}
