\definecolor{brightube}{rgb}{0.82, 0.62, 0.91}

\colorlet{BaseColor}{NavyBlue}
\colorlet{PrefixColor}{BrickRed}
\colorlet{SuffixColor}{Emerald}
\colorlet{RootColor}{CadetBlue}
\colorlet{EndingColor}{OliveGreen}
\colorlet{PostfixColor}{brightube}

\newcommand{\MorphemeSignYPad}{2pt}
\newcommand{\MorphemeSignXPad}{1pt}
\newcommand{\MorphemeSignHeight}{4pt}
\newcommand\MorphemeSignLineWidth{1pt}



\newcommand{\RSMorphemeSignHeight}{4}
\newcommand{\RSMorphemeSignXPad}{1}
\newcommand{\RSMorphemeSignYPad}{3}
\newcommand{\RSMorphemeSignLineWidth}{2}

\newlength{\RSLenTallAux}
\newlength{\RSLenNormalAux}
\settoheight\RSLenNormalAux{о}    
\getlength{\RSLenNormal}{\RSLenNormalAux}

\NewDocumentCommand{\rsGetMorphemeShift}{ m }{%
    \settoheight\RSLenTallAux{#1}%
    \getlength{\RSLenTall}{\RSLenTallAux}%
    \FPsub\RSMorphemeSignYPadAux\RSLenTall\RSLenNormal%
    \FPifneg\RSMorphemeSignYPadAux%
        \FPSet\RSMorphemeSignYPadAux{0}%
    \else%
    \fi%
}

\tikzset{%
    RSMorphemeLine/.style={%
        line width=\RSMorphemeSignLineWidth pt,
    },
}



\newlength{\LenBig}%
\newlength{\LenSmall}%
\settoheight\LenSmall{а}%
\getlength{\TmpSmall}{\LenSmall}%

\NewDocumentCommand{\ExamineMorphemeShift}{ m }{%
    \settoheight\LenBig{#1}%
    \getlength{\TmpBig}{\LenBig}%
    \FPsub\MorphemeYShiftAux\TmpBig\TmpSmall%
    \FPifneg\MorphemeYShiftAux%
        \FPset\MorphemeYShiftAux{0}%
    \else%
    \fi%
}


\ExplSyntaxOn

\keys_define:nn { rs/morphemes/phantom }{%
    phantom .bool_set:N = \l_rs_morphemes_phantom_bool,
    phantom .default:n = true
}

\ExplSyntaxOff


% ====================================================================
% ======================== Prefix (приставка) ========================
% ====================================================================

\ExplSyntaxOn

    \NewDocumentCommand{\rsPrefixAux}{ s O{} m }{%
        \rsGetMorphemeShift{#3}
        \ifblank{#3}{}{%
            \keys_set:nn { rs/morphemes/phantom } {%
                phantom=false, % initialize to false
                #2
             }
            \tikz[baseline=(Root.base)]{%
                \bool_if:NTF \l_rs_morphemes_phantom_bool{
                    \begingroup
                        \StrLen{#3}[\tmp]
                        \node[RSCage] (Root) {\phantom{\dorepeate{\tmp}{о}}};
                        \node at (Root.center) {\textbullet};
                    \endgroup
                }{
                    \node[RSCage] (Root) {#3};
                }
                \draw[
                    RSMorphemeLine,
                    \IfBooleanTF{#1}{RSPrefixColor}{black},
                    transform~canvas={%
                        yshift=\RSMorphemeSignYPad - \RSMorphemeSignYPadAux pt,
                    },
                ]
                    let
                        \p1 = (Root.north~west),
                        \p2 = (Root.north~east)
                    in
                        (\x1 + \RSMorphemeSignXPad, \y1 + \RSMorphemeSignHeight) --
                        (\x2 - \RSMorphemeSignXPad, \y2 + \RSMorphemeSignHeight) --
                        ++(0, -\RSMorphemeSignHeight pt);
            }%
        }%
    }
    
    \NewDocumentCommand{\rsPrefix}{ s O{} m }{%
        \keys_set:nn { rs/morphemes/phantom } {%
            phantom=false, % initialize to false
            #2
         }
        \ifblank{#3}{}{%
            \IfBooleanTF{#1}{%
                \bool_if:NTF \l_rs_morphemes_phantom_bool{%
                    \ProcessVector{\rsPrefixAux*[phantom]}{#3}%
                }{%
                    \ProcessVector{\rsPrefixAux*}{#3}%
                }%
            }{%
                \bool_if:NTF \l_rs_morphemes_phantom_bool{%
                    \ProcessVector{\rsPrefixAux[phantom]}{#3}%
                }{
                    \ProcessVector{\rsPrefixAux}{#3}%
                }%
            }%
        }%
    }
\ExplSyntaxOff


% ====================================================================
% ======================= Postfix (постфикс) =========================
% ====================================================================

    \NewDocumentCommand{\WordPostfixAux}{ m O{black} } {%
        \ifblank{#1}{}%
        {%
            \tikz[baseline=(Root.base)]{
                \node[
                    inner sep=0pt,
                    outer sep=0pt
                ] (Root) {#1};
    
                \draw[
                    #2,
                    line width=\MorphemeSignLineWidth
                ]
                    let
                        \p1 = (Root.north west),
                        \p2 = (Root.north east)
                    in
                        (\x1 + \MorphemeSignXPad, \y1 + \MorphemeSignYPad) --
                        (0.5 * \x1 + 0.5 * \x2, \y2 + \MorphemeSignYPad + \MorphemeSignHeight) -- 
                        (\x2 - \MorphemeSignXPad, \y2 + \MorphemeSignYPad)
                        (\x1 + \MorphemeSignXPad, \y1 + \MorphemeSignYPad + 3pt) --
                        (0.5 * \x1 + 0.5 * \x2, \y2 + \MorphemeSignYPad + \MorphemeSignHeight + 3pt) -- 
                        (\x2 - \MorphemeSignXPad, \y2 + \MorphemeSignYPad + 3pt)
                        ;
            }%
        }%
    }


    \NewDocumentCommand{\WordPostfix}{ s m }{%
        \IfBooleanTF{#1}{\WordPostfixAux{#2}[PostfixColor]}{\WordPostfixAux{#2}}%
    }
    
    
    \NewDocumentCommand{\WordPostfixSeq}{ s m }{%
        \ifblank{#2}{}{%
            \IfBooleanTF{#1}{%
                \ProcessVector{\WordPostfix*}{#2}%
                }{\ProcessVector{\WordPostfix}{#2}}%
        }%
    }


% ====================================================================
% ======================== Suffix (суффикс) ==========================
% ====================================================================

\ExplSyntaxOn

\NewDocumentCommand{\rsSuffixAux}{ s O{} m }{%
    \ifblank{#3}{}{%
        \keys_set:nn { rs/morphemes/phantom } {%
            phantom=false, % initialize to false
            #2
         }
        \tikz[baseline=(Root.base)]{%
            \bool_if:NTF \l_rs_morphemes_phantom_bool{
                \begingroup
                    \StrLen{#3}[\tmp]
                    \node[RSCage] (Root) {\phantom{\dorepeate{\tmp}{о}}};
                    \node at (Root.center) {\textbullet};
                \endgroup
            }{
                \node[RSCage] (Root) {#3};
            }
            \draw[
                RSMorphemeLine,
                \IfBooleanTF{#1}{RSSuffixColor}{black},
            ]
                let
                    \p1 = (Root.north~west),
                    \p2 = (Root.north~east)
                in
                    (\x1 + \RSMsignXPad, \y1 + \RSMsignYPad) --
                    (.5 * \x1 + .5 * \x2, \y2 + \RSMsignYPad + \RSMsignHeight) --
                    (\x2 - \RSMsignXPad, \y2 + \RSMsignYPad);
        }%
    }%
}

\NewDocumentCommand{\rsSuffix}{ s O{} m }{%
    \keys_set:nn { rs/morphemes/phantom } {%
        phantom=false, % initialize to false
        #2
     }
    \ifblank{#3}{}{%
        \IfBooleanTF{#1}{%
            \bool_if:NTF \l_rs_morphemes_phantom_bool{%
                \ProcessVector{\rsSuffixAux*[phantom]}{#3}%
            }{%
                \ProcessVector{\rsSuffixAux*}{#3}%
            }%
        }{%
            \bool_if:NTF \l_rs_morphemes_phantom_bool{%
                \ProcessVector{\rsSuffixAux[phantom]}{#3}%
            }{
                \ProcessVector{\rsSuffixAux}{#3}%
            }%
        }%
    }%
}

\ExplSyntaxOff



% ====================================================================
% ========================== Root (корень) ===========================
% ====================================================================


    \NewDocumentCommand{\WordRootAux}{ m O{black} }{%
        \ExamineMorphemeShift{#1}%
        \tikz[baseline=(Root.base)]{%
            \node[inner sep=0pt, outer sep=0pt] (Root) {#1};%
            
            \draw[%
                #2,
                line width=\MorphemeSignLineWidth,%
                transform canvas={
                    yshift=\MorphemeSignYPad,
                },
            ]%
                ([xshift=\MorphemeSignXPad]Root.north west)%
                parabola[bend pos=0.5] bend + (0, \MorphemeSignHeight)%
                ([xshift=-\MorphemeSignXPad]Root.north east);%
        }%
    }
    
    \NewDocumentCommand{\WordRoot}{ s m }{%
        \IfBooleanTF{#1}{\WordRootAux{#2}[RootColor]}{\WordRootAux{#2}}%
    }
    
    
    \NewDocumentCommand{\WordRootSeq}{ s m }{%
        \IfBooleanTF{#1}{%
            \ProcessVector{\WordRoot*}{#2}%
            }{\ProcessVector{\WordRoot}{#2}}%
    }



% ====================================================================
% ======================== Ending (окончание) ========================
% ====================================================================

    \NewDocumentCommand{\WordEndingTmp}{ m O{black}}{%
        \tikz[baseline=(Root.base)]{%
            \node[
                inner sep=1pt,
                outer sep=0pt
            ] (Root) {%
                \fullVPhantom\ifblank{#1}{\hphantom{o}}{#1}%
            };%
            \draw[
                #2,
                line width=\MorphemeSignLineWidth,
            ]%
                let%
                    \p1 = (Root.south west),%
                    \p2 = (Root.north east)%
                in%
                    (\x1, \y1 - \MorphemeSignYPad + 1) |-%
                    (\x2, \y2 + \MorphemeSignYPad) |-%
                    cycle;%
        }%
    }
    
    \NewDocumentCommand{\WordEnding}{ s m }{%
        \IfBooleanTF{#1}{\WordEndingTmp{#2}[EndingColor]}{\WordEndingTmp{#2}}%
    }


% ====================================================================
% =========================== Base (основа) ==========================
% ====================================================================
    
        \NewDocumentCommand{\WordBaseTmp}{ m O{black} }{%
            \IfValueT{#1}{%
                \tikz[baseline=(Root.base)]{%
                    \node[inner sep=0pt, outer sep=0pt] (Root) {#1\downVPhantom};
                    \draw[#2, line width=\MorphemeSignLineWidth]
                        let
                            \p1 = (Root.south west),
                            \p2 = (Root.south east)
                        in
                            (\x1 + 1.5 * \MorphemeSignXPad, \y1 + .15 * \MorphemeSignYPad) --
                            ++(0, -1.15 * \MorphemeSignYPad) -|
                            (\x2 - 1.5 * \MorphemeSignXPad, \y2 + .15 * \MorphemeSignYPad);
                }%
            }%
        }
        
        \NewDocumentCommand{\WordBase}{ s m }{%
            \IfBooleanTF{#1}{\WordBaseTmp{#2}[BaseColor]}{\WordBaseTmp{#2}}%
        }
        
        
    % Right base (Правая основа). 
        \NewDocumentCommand{\WordBaseRTmp}{ m O{black} }{%
            \IfValueT{#1}{%
                \tikz[baseline=(Root.base)]{%
                    \node[inner sep=0pt, outer sep=0pt] (Root) {#1\downVPhantom};
                    \draw[#2, line width=\MorphemeSignLineWidth]
                        let
                            \p1 = (Root.south west),
                            \p2 = (Root.south east)
                        in
                            (\x1 + \MorphemeSignXPad, \y1 - \MorphemeSignYPad) -|
                            (\x2 - \MorphemeSignXPad, \y2 + .15 * \MorphemeSignYPad);
                }%
            }%
        }
        
        \NewDocumentCommand{\WordBaseR}{ s m }{%
            \IfBooleanTF{#1}{\WordBaseRTmp{#2}[BaseColor]}{\WordBaseRTmp{#2}}%
        }
        
        
    % Left base (Левая основа). 
        \NewDocumentCommand{\WordBaseLTmp}{ m O{black} }{%
            \IfValueT{#1}{%
                \tikz[baseline=(Root.base)]{%
                    \node[inner sep=0pt, outer sep=0pt] (Root) {#1\downVPhantom};
                    \draw[
                        #2,
                        line width=\MorphemeSignLineWidth
                    ]
                        let
                            \p1 = (Root.south west),
                            \p2 = (Root.south east)
                        in
                            (\x1 + \MorphemeSignXPad, \y1 + .15 * \MorphemeSignYPad) --
                            ++(0, -1.15 * \MorphemeSignYPad pt) --
                            (\x2 - \MorphemeSignXPad, \y2 - \MorphemeSignYPad);
                }%
            }%
        }
        
        \NewDocumentCommand{\WordBaseL}{ s m }{%
            \IfBooleanTF{#1}{\WordBaseLTmp{#2}[BaseColor]}{\WordBaseLTmp{#2}}%
        }


% ====================================================================
% ================ Word Analysis (стандартный разбор) ================
% ====================================================================
% Стандартный морфмемный разбор слова (основа непрерывна).
    % \NewDocumentCommand{\StdWordAnalysis}{ smmmmm }{%
    %     \IfBooleanTF{#1}{%
    %         \WordBase*{%
    %             \WordPrefixSeq*{#2}%
    %             \WordRootSeq*{#3}%
    %             \WordSuffixSeq*{#4}%
    %         }\WordEnding*{#5}%
    %         \WordPostfixSeq*{#6}%
    %     }{%
    %         \WordBase{%
    %             \WordPrefixSeq{#2}%
    %             \WordRootSeq{#3}%
    %             \WordSuffixSeq{#4}%
    %         }\WordEnding{#5}%
    %         \WordPostfixSeq{#6}%
    %     }%
    % }