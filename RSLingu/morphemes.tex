\definecolor{RSSuffixColor}{HTML}{7fdb6a}
\definecolor{RSPrefixColor}{HTML}{f44747}
\definecolor{RSRootColor}{HTML}{eedc31}
\definecolor{RSPostfixColor}{HTML}{966842}
\definecolor{RSEndingColor}{HTML}{0e68ce}
\definecolor{RSBaseColor}{HTML}{c0c0c0}


\newcommand{\RSMorphemeSignHeight}{4}
\newcommand{\RSMorphemeSignXPad}{1.5}
\newcommand{\RSMorphemeSignYPad}{1}
\newcommand{\RSMorphemeSignLineWidth}{1.5}

\newcommand{\RSBaseSignXPad}{1}
\newcommand{\RSBaseSignYPad}{1}
\newcommand{\RSBaseSignHeight}{2.5}

\newcommand{\RSPhantomBool}{0}
\newcommand{\RSColorBool}{0}

\newlength{\RSLenTallAux}
\newlength{\RSLenNormalAux}
\settoheight\RSLenNormalAux{о}    
\getlength{\RSLenNormal}{\RSLenNormalAux}

\FPset\RSMorphemeSignYPadAux{0}%

\tikzset{%
    RSMorphemeLine/.style={%
        line width=\RSMorphemeSignLineWidth pt,
    },
    RSMorphemeLineShifted/.style={%
        RSMorphemeLine,
        transform canvas={%
            yshift=\RSMorphemeSignYPad - \RSMorphemeSignYPadAux pt,
        },
    }
}




% +++++++++++++++++++++++++++++++++++++++++++++++++++++++++++++++++++++++++++++++++++
% +++++++++++++++++++++++++++++++++++ FUNCTIONAL ++++++++++++++++++++++++++++++++++++
% +++++++++++++++++++++++++++++++++++++++++++++++++++++++++++++++++++++++++++++++++++

% Functional.
\ExplSyntaxOn

    % Calculate height difference between input phrase and «о» symbol.
    \cs_new_protected:Nn \l__rs_get_morpheme_sign_shift:n {
        \settoheight\RSLenTallAux{#1}
        \getlength{\RSLenTall}{\RSLenTallAux}
        \FPsub\RSMorphemeSignYPadAux\RSLenTall\RSLenNormal
        \FPifneg\RSMorphemeSignYPadAux
            \FPset\RSMorphemeSignYPadAux{0}
        \else
        \fi
    }
    \cs_generate_variant:Nn \l__rs_get_morpheme_sign_shift:n { N }


    % For `phantom` option.
    \cs_new_protected:Nn \l__rs_choose_text:n {
        \bool_if:nTF { \RSPhantomBool || \l_rs_morphemes_phantom_bool } {
            \group_begin:
                \StrLen{#1}[\l__tmp]
                \node[RSCage] (Root) {\phantom{\dorepeate{\l__tmp}{о}}\RSUpperVPhantom};
                \node at (Root.center) {\textbullet};
            \group_end:
        }{
            \node[RSCage] (Root) {#1\RSUpperVPhantom};
        }
    }
    \cs_generate_variant:Nn \l__rs_choose_text:n { N }
    
    % For `color` option.
    \tl_new:N \l__rs_color
    \cs_new_protected:Nn \l__rs_choose_color:n {
        \tl_set:Nn \l__rs_color { \bool_if:nTF {\RSColorBool || \l_rs_morphemes_color_bool}{#1}{black} }
    }
    
    % Morphemes key-val arguments.
    \keys_define:nn { rs/morphemes/keys }{%
        phantom .bool_set:N = \l_rs_morphemes_phantom_bool,
        phantom .default:n = true,
        color .bool_set:N = \l_rs_morphemes_color_bool,
        color .default:n = true
    }
    
    % Base key-val arguments.
    \keys_define:nn { rs/morphemes/base/keys }{%
        left .bool_set:N = \l_rs_base_left_bool,
        left .default:n = true,
        right .bool_set:N = \l_rs_base_right_bool,
        right .default:n = true
    }

    % Environment.
    \NewDocumentEnvironment{rslingu}{ O{} }
    {
        \keys_set:nn { rs/morphemes/keys } { phantom=false, color=false, #1 }
        \bool_if:NTF \l_rs_morphemes_phantom_bool { \renewcommand{\RSPhantomBool}{1} }{}
        \bool_if:NTF \l_rs_morphemes_color_bool { \renewcommand{\RSColorBool}{1} }{}
    }
    {
        \renewcommand{\RSPhantomBool}{0}
        \renewcommand{\RSColorBool}{0}
    }

\ExplSyntaxOff



% +++++++++++++++++++++++++++++++++++++++++++++++++++++++++++++++++++++++++++++++++++
% ++++++++++++++++++++++++++++++++ MORPHEME COMMANDS ++++++++++++++++++++++++++++++++
% +++++++++++++++++++++++++++++++++++++++++++++++++++++++++++++++++++++++++++++++++++

% ======================================================
% ================ Prefix (приставка) ==================
% ======================================================

\ExplSyntaxOn

    \NewDocumentCommand{\rsPrefix}{ O{} m }{
        \group_begin:
        \IfNoValueTF{#2}{}{
            \keys_set:nn { rs/morphemes/keys } { phantom=false, color=false, #1 }

            \clist_map_variable:nNn { #2 } { \l__item } {
                \tikz[baseline=(Root.base)]{
                    \l__rs_choose_text:N \l__item
                    
                    \l__rs_get_morpheme_sign_shift:N \l__item
                    \l__rs_choose_color:n { RSPrefixColor }
                    \draw[
                        RSMorphemeLineShifted,
                        \l__rs_color,
                    ]
                        let
                            \p1 = (Root.north~west),
                            \p2 = (Root.north~east)
                        in
                            (\x1 + \RSMorphemeSignXPad, \y1 + \RSMorphemeSignHeight) --
                            (\x2 - \RSMorphemeSignXPad, \y2 + \RSMorphemeSignHeight) --
                            ++(0, - \RSMorphemeSignHeight pt);
                }
            }
        }
       \group_end:
    }

\ExplSyntaxOff

% -----------------------------------------------------
% -----------------------------------------------------




% ======================================================
% ================== Root (корень) =====================
% ======================================================

\ExplSyntaxOn

    \NewDocumentCommand{\rsRoot}{ O{} m }{
        \group_begin:
        \IfNoValueTF{#2}{}{
            \keys_set:nn { rs/morphemes/keys } { phantom=false, color=false, #1 }

            \clist_map_variable:nNn { #2 } { \l__item } {
                \tikz[baseline=(Root.base)]{
                    \l__rs_choose_text:N \l__item
                    
                    \l__rs_get_morpheme_sign_shift:N \l__item
                    \l__rs_choose_color:n { RSRootColor }
                    \draw[
                        RSMorphemeLineShifted,
                        \l__rs_color,
                    ]
                        let
                            \p1 = (Root.north~west),
                            \p2 = (Root.north~east)
                        in
                            (\x1 + \RSMorphemeSignXPad, \y1 + \RSMorphemeSignYPad)
                            parabola[bend~pos=.5] bend + (0, \RSMorphemeSignHeight pt)
                            (\x2 - \RSMorphemeSignXPad, \y2 + \RSMorphemeSignYPad);
                }
            }
        }
       \group_end:
    }

\ExplSyntaxOff
% -----------------------------------------------------
% -----------------------------------------------------




% ======================================================
% ================= Suffix (суффикс) ===================
% ======================================================
\ExplSyntaxOn

    \NewDocumentCommand{\rsSuffix}{ O{} m }{
        \group_begin:
        \IfNoValueTF{#2}{}{
            \keys_set:nn { rs/morphemes/keys } { phantom=false, color=false, #1 }

            \clist_map_variable:nNn { #2 } { \l__item } {
                \tikz[baseline=(Root.base)]{
                    \l__rs_choose_text:N \l__item

                    \l__rs_choose_color:n { RSSuffixColor }
                    \draw[
                        RSMorphemeLineShifted,
                        \l__rs_color,
                    ]
                        let
                            \p1 = (Root.north~west),
                            \p2 = (Root.north~east)
                        in
                            (\x1 + \RSMorphemeSignXPad, \y1 + \RSMorphemeSignYPad) --
                            (.5 * \x1 + .5 * \x2, \y2 + \RSMorphemeSignYPad + \RSMorphemeSignHeight) --
                            (\x2 - \RSMorphemeSignXPad, \y2 + \RSMorphemeSignYPad);
                }
            }
        }
       \group_end:
    }

\ExplSyntaxOff
% -----------------------------------------------------
% -----------------------------------------------------




% ================= Postfix (постфикс) =================
% ======================================================
\ExplSyntaxOn

    \NewDocumentCommand{\rsPostfix}{ O{} m }{
        \group_begin:
        \IfNoValueTF{#2}{}{
            \keys_set:nn { rs/morphemes/keys } { phantom=false, color=false, #1 }

            \clist_map_variable:nNn { #2 } { \l__item } {
                \tikz[baseline=(Root.base)]{
                    \l__rs_choose_text:N \l__item

                    \l__rs_choose_color:n { RSPostfixColor }
                    \draw[
                        RSMorphemeLineShifted,
                        \l__rs_color,
                    ]
                        let
                            \p1 = (Root.north~west),
                            \p2 = (Root.north~east)
                        in
                            (\x1 + \RSMorphemeSignXPad, \y1 + \RSMorphemeSignYPad) --
                            (.5 * \x1 + .5 * \x2, \y2 + \RSMorphemeSignYPad + \RSMorphemeSignHeight) --
                            (\x2 - \RSMorphemeSignXPad, \y2 + \RSMorphemeSignYPad)
                            (\x1 + \RSMorphemeSignXPad, \y1 + \RSMorphemeSignYPad + 3pt) --
                            (.5 * \x1 + .5 * \x2, \y2 + \RSMorphemeSignYPad + \RSMorphemeSignHeight + 3pt) --
                            (\x2 - \RSMorphemeSignXPad, \y2 + \RSMorphemeSignYPad + 3pt);
                }
            }
        }
       \group_end:
    }

\ExplSyntaxOff
% -----------------------------------------------------
% -----------------------------------------------------




% ======================================================
% ================= Ending (окончание) =================
% ======================================================
\ExplSyntaxOn

    \NewDocumentCommand{\rsEnding}{ O{} m }{
        \group_begin:
        \keys_set:nn { rs/morphemes/keys } { phantom=false, color=false, #1 }

        \tikz[baseline=(Root.base)]{
            
            \bool_if:nTF { \RSPhantomBool || \l_rs_morphemes_phantom_bool }
            {
                \node[RSCage] (Root) {\phantom{о}\RSFullVPhantom};
                \node at (Root.center) {\textbullet};
            }{
                \node[RSCage] (Root) {\IfNoValueTF{#2}{#2}{\phantom{о}}\RSFullVPhantom};
            }

            \l__rs_choose_color:n { RSEndingColor }
            \draw[
                RSMorphemeLine,
                \l__rs_color,
            ]
                let
                    \p1 = (Root.south~west),
                    \p2 = (Root.north~east)
                in
                    (\x1 - 2, \y1 - \RSBaseSignYPad - \RSBaseSignHeight) |-%
                    (\x2 + 2, \y2 + \RSMorphemeSignYPad + \RSMorphemeSignHeight pt ) |-%
                    cycle;%
        }
      \group_end:
    }

\ExplSyntaxOff
% -----------------------------------------------------
% -----------------------------------------------------




% ======================================================
% =================== Base (основа) ====================
% ======================================================
\ExplSyntaxOn

    \NewDocumentCommand{\rsBase}{ O{} m }{
        \group_begin:
        \IfNoValueTF{#2}{}{
            \keys_set_known:nn { rs/morphemes/keys } { color=false, #1 }
            \keys_set_known:nn { rs/morphemes/base/keys } { right=false, left=false, #1 }

            \tikz[baseline=(Root.base)]{
                \node[RSCage] (Root) {#2\RSDownVPhantom};

                \l__rs_choose_color:n { RSBaseColor }
                \FPdiv\RSAux\RSMorphemeSignLineWidth{2}
                \foreach \n in {0}
                    \draw[
                        RSMorphemeLine,
                        \l__rs_color,
                        transform~canvas={
                            yshift=-\RSBaseSignYPad - \RSBaseSignHeight pt,
                        },
                    ]
                        let
                            \p1 = ([xshift=\RSBaseSignXPad pt]Root.south~west),
                            \p2 = ([xshift=-\RSBaseSignXPad - .5 pt]Root.south~east)
                        in
                            (\p1) -- (\p2)
                            \bool_if:NTF \l_rs_base_right_bool {}
                                {([yshift=-\RSAux pt]\p1) -- ([yshift=\RSBaseSignHeight pt]\p1)}
                            \bool_if:NTF \l_rs_base_left_bool {}
                                {([yshift=-\RSAux pt]\p2) -- ([yshift=\RSBaseSignHeight pt]\p2)};
            }
        }
      \group_end:
    }

\ExplSyntaxOff
% -----------------------------------------------------
% -----------------------------------------------------




\ExplSyntaxOn

    % Morphemic Analysis.
    \NewDocumentCommand{\rsMorphemicAnalysis}{ O{} mmmmm }{
        \begin{rslingu}[#1]
            \rsBase{\rsPrefix{#2}\rsRoot{#3}\rsSuffix{#4}}\rsEnding{#5}\rsPostfix{#6}
        \end{rslingu}
    }

\ExplSyntaxOff