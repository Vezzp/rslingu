\documentclass[20pt, dvipsnames]{extarticle}

\usepackage[bottom=2cm,
            left=2cm,
            right=2cm,
            top=2cm]{geometry}

\linespread{1.5}

\usepackage{ifxetex}
\ifxetex
    \usepackage[cm-default]{fontspec}
    \defaultfontfeatures{Ligatures=TeX}
    \usepackage{polyglossia}
    \setdefaultlanguage{russian}
    \setmainfont{Times New Roman}
    \newfontfamily\cyrillicfont{Times New Roman}[Script=Cyrillic]
\else
   \usepackage[utf8]{inputenc}
   \usepackage[russian]{babel}
\fi

\usepackage{RSLingu/rslingu}


\begin{document}

Начинайте набирать текст.

\rsAttribute*{привет}
\rsObject*{привет}
\rsSubject*{пивет}
\rsAdverbarial*{привет}
\rsPredicate*{привет}


\rsAttribute{привет}
\rsObject{привет}
\rsSubject{пивет}
\rsAdverbarial{привет}
\rsPredicate{привет}

\rsNoun{дом}
\rsNoun[ед.ч.]{дом}
\rsPronoun{вашего}
\rsVerb[инф.]{ходить}
\rsUnion{и}



% #966842	(150,104,66)
% #f44747	(244,71,71)
% #eedc31	(238,220,49)
% #7fdb6a	(127,219,106)
% #0e68ce	(14,104,206)

\newcommand{\RSMlineWidth}{1.5}
\newcommand{\RSMsignXPad}{1.5}
\newcommand{\RSMsignYPad}{1.5}
\newcommand{\RSMsignHeight}{4}


\tikzset{
    RSMlineWidth/.style={
        line width=\RSMlineWidth pt,
    }
}


\definecolor{RSSuffixColor}{HTML}{966842}
\definecolor{RSPrefixColor}{HTML}{f44747}


\rsSuffixAux{суффикс}
\rsSuffixAux*[phantom]{суффикс}
\rsSuffix{суф, фикс}
\rsSuffix*{суф, фикс}
\rsSuffix*[phantom]{суф, фикс}


\rsPrefixAux{приставка}
\rsPrefixAux*{приставка}
\rsPrefix{при, став, ка}
\rsPrefix*{при, став, ка}
\rsPrefix[phantom]{при, став, ка}
\rsPrefix*[phantom]{при, став, ка}

\end{document}